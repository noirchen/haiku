%!TEX root = haiku.tex
\book{\LARGE \FK 日本俳句史}

\newpage

\chapter{\FK 古今俳句佳作一千首}

\setcounter{haikucounter}{0}

\begin{haiku}
    {\FH 風さむし破れ\ruby{障子}{しょうじ}の\ruby{神無月}{かんなづき}}\hfill{\FH 宗鑑}

    {\FK 十月纸窗破遂识金风寒}
\end{haiku}

\begin{haiku}
    {\FH 月に\ruby{柄}{え}をさしたらばよき\ruby{団扇}{うちわ}かな}\hfill{\FH 同}

    {\FK 良月若安柄绝似佳团扇}
\end{haiku}

\begin{haiku}
    {\FH \ruby{青柳}{あおやぎ}の眉かく岸のひたひかな}\hfill{\FH 守武}

    {\FK 河岸似前额青柳写双眉}
\end{haiku}

\begin{haiku}
    {\FH 落花枝にかえるとみれば胡蝶かな}\hfill{\FH 同}

    {\FK 蝴蝶翩翩舞落花疑返枝}
\end{haiku}

\begin{haiku}
    {\FH 蝶々のように梢に舞う落花}\hfill{\FH 同}

    {\FK 翻飞似蛱蝶落花舞梢头}
\end{haiku}

\begin{haiku}
    {\FH 春と夏と秋と今日とのしぐれかな}\hfill{\FH 貞德}

    {\FK 今日又时雨还同春夏秋}
\end{haiku}

\begin{haiku}
    {\FH 月影を\ruby{汲}{く}みこぼしけり\ruby{手水鉢}{ちょうずばち}}\hfill{\FH 立圃}

    {\FK 月影今满溢小钵汲水来}
\end{haiku}

\begin{haiku}
    {\FH 海棠もつれてる睡る蝴蝶かな}\hfill{\FH 同}

    {\FK 而今共入梦蝴蝶偕海棠}
\end{haiku}

\begin{haiku}
    {\FH そちは何をなげきの森の夜の声}\hfill{\FH 貞室}

    {\FK 何物感怀今叹惋彼处森林夜作声}
\end{haiku}

\begin{haiku}
    {\FH うぐいすや\ruby{国栖}{くず}の翁の笛の弟子}\hfill{\FH 同}

    {\FK 国栖老翁正课徒教渠吹笛莺声中}
\end{haiku}

\begin{haiku}
    {\FH 夏痩とこたえて後は涙かな}\hfill{\FH 季吟}

    {\FK 人因炎夏瘦答后双泪流}
\end{haiku}

\begin{haiku}
    {\FH まざまざと\ruby{在}{いま}すが如し魂まつり}\hfill{\FH 同}

    {\FK 历历人如在亡魂祭奠中}
\end{haiku}

\begin{haiku}
    {\FH 年玉に梅折る小野の翁かな}\hfill{\FH 言水}

    {\FK 折梅为年礼野居一老翁}
\end{haiku}

\begin{haiku}
    {\FH \ruby{阿蘭陀}{おらんだ}の文字か橫たふ\ruby{天津}{あまつ}雁}\hfill{\FH 宗因}

    {\FK 有如荷兰字候雁书长空}
\end{haiku}

\begin{haiku}
    {\FH \ruby{秀}{ひい}でたる詞の花はこれや蘭}\hfill{\FH 同}

    {\FK 幽兰竟何似秀丽称词花}
\end{haiku}

\begin{haiku}
    {\FH 書初や行年七十摂州の住}\hfill{\FH 同}

    {\FK 新岁挥毫初书字行年七十住摄州}
\end{haiku}

\begin{haiku}
    {\FH 河上や宮前の楊柳眼前の花}\hfill{\FH 同}

    {\FK 河上风光一望好宫前杨柳眼前花}
\end{haiku}

\begin{haiku}
    {\FH 古池や蛙とびこむ水の音}\hfill{\FH 芭蕉}

    {\FK 蛙跃古池内静潴传清响}
\end{haiku}

\begin{haiku}
    {\FH 枯枝に烏とまりけり秋のくれ}\hfill{\FH 同}

    {\FK 秋日今向暮枯枝有乌栖}
\end{haiku}

\begin{haiku}
    {\FH 夏草やつわものどもが夢のあと}\hfill{\FH 同}

    {\FK 大藩歌舞地曾作修罗场今看夏草盛功名等黄粱}
\end{haiku}

\begin{haiku}
    {\FH 五月雨をあつめて早し最上川}\hfill{\FH 同}

    {\FK 齐集五月雨奔腾最上川}
\end{haiku}

\begin{haiku}
    {\FH 川舟やよい茶よい酒よい月夜}\hfill{\FH 同}

    {\FK 中流荡舟寄逸兴茶甘酒美月有情}
\end{haiku}

\begin{haiku}
    {\FH 病雁の夜寒に落ちてたびねかな}\hfill{\FH 同}

    {\FK 长空病雁落旅宿觉夜寒}
\end{haiku}

\begin{haiku}
    {\FH ふるさとや臍の緒に泣く年のくれ}\hfill{\FH 同}

    {\FK 云游返故里睹吾旧脐带况复岁云暮泪下吞声哀}
\end{haiku}

\begin{haiku}
    {\FH 物言えば唇さむし秋の風}\hfill{\FH 同}

    {\FK 张口欲有诉秋风吹唇寒}
\end{haiku}

\begin{haiku}
    {\FH つかもうごけ我泣く声は秋の風}\hfill{\FH 同}

    {\FK 悼一笑}

    {\FK 恍惚疑塚动哭君秋风中}
\end{haiku}

\begin{haiku}
    {\FH この道や行く人なしに秋の暮}\hfill{\FH 同}

    {\FK 无人行此道凉秋暮色中}
\end{haiku}

\begin{haiku}
    {\FH 降らずとも竹植うる日は蓑と笠}\hfill{\FH 同}

    {\FK 此日种竹去五月幽兴多夏雨纵不降戴笠复被蓑}
\end{haiku}

\begin{haiku}
    {\FH 埋火やかべには客の影法師}\hfill{\FH 同}

    {\FK 炭火埋灰里客影壁上游}
\end{haiku}

\begin{haiku}
    {\FH 荒海や佐渡によこたう天の川}\hfill{\FH 同}

    {\FK 海涛正汹涌银汉横佐渡}
\end{haiku}

\begin{haiku}
    {\FH ありがたやいただいてふむ橋の霜}\hfill{\FH 同}

    {\FK 祝深川大桥落成}

    {\FK 践踏夜霜过感激度此桥}
\end{haiku}

\begin{haiku}
    {\FH 我が宿は蚊の小さきを馳走かな}\hfill{\FH 同}

    {\FK 陋室无长物小蚊款嘉宾}
\end{haiku}

\begin{haiku}
    {\FH 埋火も消ゆや涙の烹る音}\hfill{\FH 同}

    {\FK 唁人丧子}

    {\FK 炭火亦已熄煎熬热泪声}
\end{haiku}

\begin{haiku}
    {\FH 一声の江に横たふやほととぎす}\hfill{\FH 同}

    {\FK 四月时鸟鸣横江传一声}
\end{haiku}

\begin{haiku}
    {\FH 名月や池をめぐりて夜もすがら}\hfill{\FH 同}

    {\FK 中秋明月好终宵绕池行}
\end{haiku}

\begin{haiku}
    {\FH 春なれや名もなき山の朝かすみ}\hfill{\FH 同}

    {\FK 春回大地气象改无名山丘笼曙霞}
\end{haiku}

\begin{haiku}
    {\FH 雲雀より空にやすらふ峠かな}\hfill{\FH 同}

    {\FK 小憩绝顶上俯视云雀低}
\end{haiku}

\begin{haiku}
    {\FH 草の葉を落つるより飛ぶ蛍かな}\hfill{\FH 同}

    {\FK 流萤翩翩舞起落草叶间}
\end{haiku}

\begin{haiku}
    {\FH うぐいすや柳の後やぶの前}\hfill{\FH 同}

    {\FK 啼莺在何处柳后或薮前}
\end{haiku}

\begin{haiku}
    {\FH 清滝や波に散り込む青松葉}\hfill{\FH 同}

    {\FK 黛色松叶落随波散清泷}
\end{haiku}

\begin{haiku}
    {\FH 声すみて北斗にひびく\ruby{砧}{きぬた}かな}\hfill{\FH 同}

    {\FK 清澄冲北斗秋夜砧杵声}
\end{haiku}

\begin{haiku}
    {\FH 汐越や鶴脛ぬれて海涼し}\hfill{\FH 同}

    {\FK 波澜湿鹤胫凉生海水中}
\end{haiku}

\begin{haiku}
    {\FH 送られつ送りつはては木曾の秋}\hfill{\FH 同}

    {\FK 君送我兮我送君往来木曾秋气深}
\end{haiku}

\begin{haiku}
    {\FH ちる柳あるじも我も鐘をきく}\hfill{\FH 同}

    {\FK 微风吹得轻柳散主客皆闻远钟声}
\end{haiku}

\begin{haiku}
    {\FH 年くれぬ笠着て草鞋\ruby{穿}{は}きながら}\hfill{\FH 同}

    {\FK 星回斗转岁云暮竹笠芒鞋尚异乡}
\end{haiku}

\begin{haiku}
    {\FH 鷹一つ見つけてうれし\ruby{伊良}{いら}\ruby{古崎}{ござき}}\hfill{\FH 同}

    {\FK 伊良古崎今日到喜见只鹰击长空}
\end{haiku}

\begin{haiku}
    {\FH さまざまの事思い出す桜かな}\hfill{\FH 同}

    {\FK 流眄樱花思悠悠万千往事上心头}
\end{haiku}

\begin{haiku}
    {\FH 菊の香や奈良には古き仏たち}\hfill{\FH 同}

    {\FK 重九奈良风光好古佛长伴菊花香}
\end{haiku}

\begin{haiku}
    {\FH 金屏の松の古さよ冬籠り}\hfill{\FH 同}

    {\FK 金屏风上绘松古冬日守舍寂寥中}
\end{haiku}

\begin{haiku}
    {\FH 旅に病んで夢は枯野をかけ廻る}\hfill{\FH 同}

    {\FK 卧病天涯人事绝梦魂长绕枯野旋}
\end{haiku}

\begin{haiku}
    {\FH 稲妻や昨日は東今日は西}\hfill{\FH 其角}

    {\FK 电光讵有定昨东今在西}
\end{haiku}

\begin{haiku}
    {\FH 子規一二の橋の夜明かな}\hfill{\FH 同}

    {\FK 桥畔天欲曙一二子规啼}
\end{haiku}

\begin{haiku}
    {\FH 声かれて猿の歯白し峰の月}\hfill{\FH 同}

    {\FK 一轮月出高峰上啼猿齿白鸣声哀}
\end{haiku}

\begin{haiku}
    {\FH 三人の声に答えよ秋の声}\hfill{\FH 同}

    {\FK 三人哭君请答语泉台寂寥闻秋声}
\end{haiku}

\begin{haiku}
    {\FH 昨日会し人や隣の魂祭}\hfill{\FH 同}

    {\FK 悼邻人}

    {\FK 昨日犹睹伊人面今看邻家奠亡魂}
\end{haiku}

\begin{haiku}
    {\FH 夢にくる母をかえすか郭公}\hfill{\FH 同}

    {\FK 亡母而今入我梦杜鹃似催高堂归}
\end{haiku}

\begin{haiku}
    {\FH 二星恨む隣のむすめ年十五}\hfill{\FH 同}

    {\FK 邻家少艾年十五遥看二星幽恨生}
\end{haiku}

\begin{haiku}
    {\FH \ruby{簾}{れん}に入って美人\ruby{馴}{な}るる燕かな}\hfill{\FH 嵐雪}

    {\FK 已经美人驯飞燕入帘来}
\end{haiku}

\begin{haiku}
    {\FH 鴨鳴くや弓矢を捨てて十余年}\hfill{\FH 去来}

    {\FK 凫鸭相呼成感慨弓矢释手十五年}
\end{haiku}

\begin{haiku}
    {\FH 光り合う二つの山の\ruby{茂}{しげ}りかな}\hfill{\FH 同}

    {\FK 两山绿树茂相映翠光浮}
\end{haiku}

\begin{haiku}
    {\FH いそがしや\ruby{沖}{おき}の時雨の真帆片帆}\hfill{\FH 同}

    {\FK 海上时雨疾征帆相背开}
\end{haiku}

\begin{haiku}
    {\FH 凩の地まで落さぬ時雨かな\footnote{\FT 此为去来原句,芭蕉将其改为「凩の地にも落さぬ時雨かな」。}}\hfill{\FH 同}

    {\FK 时雨不坠地为有疾风吹}
\end{haiku}

\begin{haiku}
    {\FH 木も石も\ruby{眼}{まなこ}に光るあつさかな}\hfill{\FH 同}

    {\FK 木石烈日下映目俱生光}
\end{haiku}

\begin{haiku}
    {\FH かゝる夜の月も見にけり\ruby{野辺送}{のべおくり}}\hfill{\FH 同}

    {\FK 中秋夜葬送猶子}

    {\FK 如此夜月下送尔赴郊原}
\end{haiku}

\begin{haiku}
    {\FH 秋風や白木の弓に弦はらん}\hfill{\FH 同}

    {\FK 白木弓弦思满引三秋金风今吹来}
\end{haiku}

\begin{haiku}
    {\FH 千貫の\ruby{剣}{つるぎ}埋めけり\ruby{苔}{こけ}の露}\hfill{\FH 同}

    {\FK 岚兰追悼}

    {\FK 沉埋名剑值千贯凉秋碧苔清露滋}
\end{haiku}

\begin{haiku}
    {\FH \ruby{幾人}{いくたり}か時雨かけぬく瀬田の橋}\hfill{\FH 丈草}

    {\FK 几人避时雨疾行濑田桥}
\end{haiku}

\begin{haiku}
    {\FH 百姓の尻からげして\ruby{喜雨}{きう}の\ruby{野路}{のじ}}\hfill{\FH 同}

    {\FK 百姓撩后襟喜雨野路中}
\end{haiku}

\begin{haiku}
    {\FH 虫の音の中に咳出す寝覚めかな}\hfill{\FH 同}

    {\FK 病中}

    {\FK 咳嗽梦惊觉人在虫声中}
\end{haiku}

\begin{haiku}
    {\FH 友ずれの舟にねつかぬ夜寒かな}\hfill{\FH 同}

    {\FK 晚泊舟相系无眠觉夜寒}
\end{haiku}

\begin{haiku}
    {\FH 野も山も雪にとられて何にもなし}\hfill{\FH 同}

    {\FK 一望无他物山野尽雪封}
\end{haiku}

\begin{haiku}
    {\FH ふりあぐる鍬の光や春の野良}\hfill{\FH 杉風}

    {\FK 阳春到田畴扬锹生光芒}
\end{haiku}

\begin{haiku}
    {\FH 管絃や落花みだるる\ruby{幕}{まく}の内}\hfill{\FH 越人}

    {\FK 幕中落花乱弦管发清音}
\end{haiku}

\begin{haiku}
    {\FH 春風に\ruby{帯}{おび}ゆるみたる寐貌哉}\hfill{\FH 同}

    {\FK 杨贵妃}

    {\FK 衣帯自弛缓人卧春风中}
\end{haiku}

\begin{haiku}
    {\FH 奈良は鹿の鳴かざるを見て戻りけり}\hfill{\FH 同}

    {\FK 呦呦未闻曾睹鹿人游奈良今归来}
\end{haiku}

\begin{haiku}
    {\FH 行く年や親に白髪を隠しけり}\hfill{\FH 同}

    {\FK 岁月东流马齿长掩藏白发长亲前}
\end{haiku}

\begin{haiku}
    {\FH 横雲の\ruby{千切}{ちぎ}れてとぶや今朝の秋}\hfill{\FH 北枝}

    {\FK 秋肃临天下横云零乱飞}
\end{haiku}

\begin{haiku}
    {\FH 夜寒さや舟の底する砂の音}\hfill{\FH 同}

    {\FK 舟底漱砂水声凉逼夜寒}
\end{haiku}

\begin{haiku}
    {\FH 欄干にのぼるや菊の影法師}\hfill{\FH 許六}

    {\FK 扶疏上栏杆黄花影动摇}
\end{haiku}

\begin{haiku}
    {\FH 清水の上から出たり春の月}\hfill{\FH 同}

    {\FK 出自清水上冉冉春月升}
\end{haiku}

\begin{haiku}
    {\FH 長き夜やいろいろに聞く虫の声}\hfill{\FH 同}

    {\FK 聆听长夜里虫声何其多}
\end{haiku}

\begin{haiku}
    {\FH \ruby{髪剃}{かみそり}や一夜に\ruby{錆}{さ}びて五月雨}\hfill{\FH 凡兆}

    {\FK 五月剃发后一夜头生锈}
\end{haiku}

\begin{haiku}
    {\FH 市中は物のにほひや夏の月}\hfill{\FH 同}

    {\FK 夏夜现明月市中物生香}
\end{haiku}

\begin{haiku}
    {\FH 朝露や\ruby{鬱金}{うこん}畠の秋の風}\hfill{\FH 同}

    {\FK 秋来遂觉朝露重凉风拂过郁金田}
\end{haiku}

\begin{haiku}
    {\FH 鷲の巣の\ruby{樟}{くす}の枯枝に日は入るぬ}\hfill{\FH 同}

    {\FK 鹫鸟营巢樟树上日影透入枯枝间}
\end{haiku}

\begin{haiku}
    {\FH 上ゆくと下来る雲や秋の天}\hfill{\FH 同}

    {\FK 秋日天高云自在上行下驰任纵横}
\end{haiku}

\begin{haiku}
    {\FH ひだるさに馴れてよく寝る霜夜かな}\hfill{\FH 惟然}

    {\FK 断炊久成习霜夜梦方酣}
\end{haiku}

\begin{haiku}
    {\FH 別るるや柿食ひながら坂の上}\hfill{\FH 同}

    {\FK 送別芭蕉翁}

    {\FK 人于此处分两道共啖涩柿上坡来}
\end{haiku}

\begin{haiku}
    {\FH 大佛のうしろに花のさかりかな}\hfill{\FH 路通}

    {\FK 樱花何烂漫大佛身后开}
\end{haiku}

\begin{haiku}
    {\FH 親もたぬ身は年々の寒さかな}\hfill{\FH 同}

    {\FK 亲朋凋零尽年年觉沍寒}
\end{haiku}

\begin{haiku}
    {\FH \ruby{去}{い}ね去ねと人に言われつ年の暮}\hfill{\FH 同}

    {\FK 凛凛岁已暮去去被人催}
\end{haiku}

\begin{haiku}
    {\FH 大切に雲こそ\ruby{包}{ぐる}め今日の月}\hfill{\FH 同}

    {\FK 珍重今宵月行云重重遮}
\end{haiku}

\begin{haiku}
    {\FH 五月闇星を見つけて\ruby{拝}{おが}みけり}\hfill{\FH 同}

    {\FK 五月梅雨沉沉暗凝望夜星参拜中}
\end{haiku}

\begin{haiku}
    {\FH 小家つづき垣根垣根の黄菊かな}\hfill{\FH 牧童}

    {\FK 茅舍相连续黄花遍墙根}
\end{haiku}

\begin{haiku}
    {\FH すいと行く水\ruby{際}{きわ}涼し飛ぶ蛍}\hfill{\FH 同}

    {\FK 流萤疾飞去水上凉意生}
\end{haiku}

\begin{haiku}
    {\FH 朝顔や昨日は五つ今日は三つ}\hfill{\FH 同}

    {\FK 牵牛入秋放昨五今成三}
\end{haiku}

\begin{haiku}
    {\FH 凩やいづこを鳴らす琵琶の湖}\hfill{\FH 同}

    {\FK 怒号声发知何处寒风吹过琵琶湖}
\end{haiku}

\begin{haiku}
    {\FH 瀬の音の二三度かはる夜寒かな}\hfill{\FH 浪化上人}

    {\FK 声变二三度濑音送夜寒}
\end{haiku}

\begin{haiku}
    {\FH 初雪や\ruby{形}{なり}さまざまの煙出し}\hfill{\FH 同}

    {\FK 轻烟姿各别大雪初纷飞}
\end{haiku}

\begin{haiku}
    {\FH ふんばりて峠を越ゆる野分哉}\hfill{\FH 同}

    {\FK 野风自迅疾翻山越岭来}
\end{haiku}

\begin{haiku}
    {\FH 子は裸父はてゝれで早苗舟}\hfill{\FH 利牛}

    {\FK 子裸父着禈棹舟植稻秧}
\end{haiku}

\begin{haiku}
    {\FH 一いろも動く物なき霜夜かな}\hfill{\FH 野水}

    {\FK 万物寂不动霜夜一色中}
\end{haiku}

\begin{haiku}
    {\FH 酒買に舟漕もどす月夜かな}\hfill{\FH 正秀}

    {\FK 回舟因沾酒良夜月无俦}
\end{haiku}

\begin{haiku}
    {\FH \ruby{終夜}{よもすがら}秋風きくや裏の山}\hfill{\FH 曾良}

    {\FK 借宿深山里终夜闻秋风}
\end{haiku}

\begin{haiku}
    {\FH 一つ葉や一葉一葉の今朝の霜}\hfill{\FH 支考}

    {\FK 一叶复一叶叶叶今朝霜}
\end{haiku}

\begin{haiku}
    {\FH 若竹や雪の重みはまだ知らず}\hfill{\FH 乙由}

    {\FK 新竹吾怜汝未解积雪威}
\end{haiku}

\begin{haiku}
    {\FH 朝顔やつるべ取られてもらい水}\hfill{\FH 千代}

    {\FK 钓桶已缠牵牛上汲井还须乞四邻}
\end{haiku}

\begin{haiku}
    {\FH 涼しさやことに\ruby{八十年}{やそぢ}の松の声}\hfill{\FH 同}

    {\FK 秋来诚觉新凉重八十年来听松声}
\end{haiku}

\begin{haiku}
    {\FH 起きてみつ寝てみつ蚊帳の広さかな\footnote{\FT 此句有说认为是元禄时代的妓女浮桥所做,但从同时代的一首川柳「お千代さん蚊帳が広けりゃ入ろうか」来看,至少千代女时代就已经把这句归于千代女了。}}\hfill{\FH 同}

    {\FK 良人已逝守孤衾起卧唯觉蚊帐宽}
\end{haiku}

\begin{haiku}
    {\FH この秋はひざに子のない月見かな}\hfill{\FH 鬼貫}

    {\FK 悼子}

    {\FK 今秋仰明月绕膝子已空}
\end{haiku}

\begin{haiku}
    {\FH 水無月や風に吹かれに故郷へ}\hfill{\FH 同}

    {\FK 吹我还故里且乘七月风}
\end{haiku}

\begin{haiku}
    {\FH 人に遁げ人に馴るるや雀の子}\hfill{\FH 同}

    {\FK 幼雀诚可喜避人复依人}
\end{haiku}

\begin{haiku}
    {\FH 遠う来る鐘のあゆみや春霞}\hfill{\FH 同}

    {\FK 初春宿石町,闻上野远钟}

    {\FK 春日彩霞起散策闻远钟}
\end{haiku}

\begin{haiku}
    {\FH 涼風や\ruby{虚空}{こくう}に満ちて松の声}\hfill{\FH 同}

    {\FK 乔松涛声发凉风满虚空}
\end{haiku}

\begin{haiku}
    {\FH 春の水所々に見ゆるかな}\hfill{\FH 同}

    {\FK 随处皆可见悠然春水长}
\end{haiku}

\begin{haiku}
    {\FH 朝も秋ゆうべも秋の暑さかな}\hfill{\FH 同}

    {\FK 残暑未敛炎威烈秋朝秋夕畏骄阳}
\end{haiku}

\begin{haiku}
    {\FH 行く水や竹に蝉鳴く\ruby{相國寺}{しょうこくじ}}\hfill{\FH 同}

    {\FK 相国寺里竹蝉鸣无根行水自悠悠}
\end{haiku}

\begin{haiku}
    {\FH おとなしき時雨をきくや高野山}\hfill{\FH 同}

    {\FK 时雨恬然落闻声高野山}
\end{haiku}

\begin{haiku}
    {\FH 樹の奥に滝の音して花や咲く}\hfill{\FH 同}

    {\FK 树丛深处景色佳悬瀑作声樱花开}
\end{haiku}

\begin{haiku}
    {\FH 去年も咲き今年も咲くや桜の木}\hfill{\FH 同}

    {\FK 樱树梢头花已著去年曾放今又开}
\end{haiku}

\begin{haiku}
    {\FH 幾秋かなぐさめかねつ母一人}\hfill{\FH 来山}

    {\FK 母孤少慰藉不知经几秋}
\end{haiku}

\begin{haiku}
    {\FH 春風や\ruby{堤}{つつみ}ごしなる牛の声}\hfill{\FH 同}

    {\FK 和煦春风至牛声越堤来}
\end{haiku}

\begin{haiku}
    {\FH 眠る蝶それともに散る牡丹かな}\hfill{\FH 同}

    {\FK 眠蝶同坠落牡丹飞散中}
\end{haiku}

\begin{haiku}
    {\FH われをつれて我影帰る月夜かな}\hfill{\FH 素堂}

    {\FK 月色皎洁夜身影伴吾归}
\end{haiku}

\begin{haiku}
    {\FH 亡き魂やたずねてくもに唱く雲雀}\hfill{\FH 也有}

    {\FK 为寻亡魂去云雀唱九霄}
\end{haiku}

\begin{haiku}
    {\FH 酒買いにやる\ruby{慈童}{じどう}あり今日の菊}\hfill{\FH 同}

    {\FK 今日菊花好沽酒托侍童}
\end{haiku}

\begin{haiku}
    {\FH 山寺に斧の\ruby{谺}{こだま}や夏木立}\hfill{\FH 同}

    {\FK 仲夏树丛凉荫密山寺斧斤传回声}
\end{haiku}

\begin{haiku}
    {\FH 山寺の春や仏に水仙花}\hfill{\FH 同}

    {\FK 阳春冉冉临山寺佛前清供水仙花}
\end{haiku}

\begin{haiku}
    {\FH 行く春や逡巡として遅桜}\hfill{\FH 蕪村}

    {\FK 绰约樱花晚徘徊暮春时}
\end{haiku}

\begin{haiku}
    {\FH 去年より又さびしいぞ秋の暮}\hfill{\FH 同}

    {\FK 秋来又夕暮萧条胜旧年}
\end{haiku}

\begin{haiku}
    {\FH 鹿鳴くや宵の雨暁の月}\hfill{\FH 同}

    {\FK 夜雨复晓月长闻鹿呦呦}
\end{haiku}

\begin{haiku}
    {\FH 柳ちり清水かれ石ところどころ}\hfill{\FH 同}

    {\FK 清水已涸竭柳散顽石浮}
\end{haiku}

\begin{haiku}
    {\FH 寒月や枯木の中の竹三竿}\hfill{\FH 同}

    {\FK 寒月照枯木三杆竹影明}
\end{haiku}

\begin{haiku}
    {\FH 椿おちて昨日の雨こぼしけり}\hfill{\FH 同}

    {\FK 山茶承宿雨花落水亦坠}
\end{haiku}

\begin{haiku}
    {\FH 我庵は東鶯西月夜}\hfill{\FH 同}

    {\FK 吾庐脱尘俗最宜幽人栖西望月色好东闻流莺啼}
\end{haiku}

\begin{haiku}
    {\FH 花茨故郷の道に似たるかな}\hfill{\FH 同}

    {\FK 翻疑故乡径花茨一望开}
\end{haiku}

\begin{haiku}
    {\FH 村百戸菊なき門も見えぬかな}\hfill{\FH 同}

    {\FK 小村虽百户家家绕菊丛}
\end{haiku}

\begin{haiku}
    {\FH 君行くや柳緑に道遠し}\hfill{\FH 同}

    {\FK 送君赴前途柳翠道路长}
\end{haiku}

\begin{haiku}
    {\FH 西吹けば東にたまる落葉かな}\hfill{\FH 同}

    {\FK 寒风自西至落叶齐聚东}
\end{haiku}

\begin{haiku}
    {\FH \ruby{路辺}{みちのべ}の\ruby{刈藻}{かるも}花さく宵の雨}\hfill{\FH 同}

    {\FK 刈藻经夜雨参差路边开}
\end{haiku}

\begin{haiku}
    {\FH 春風や堤長うして家遠し}\hfill{\FH 同}

    {\FK 路远家难到长堤步春风}
\end{haiku}

\begin{haiku}
    {\FH 小舟にて\ruby{僧都}{そうず}送るや春の水}\hfill{\FH 同}

    {\FK 随缘送僧正小舟春水中}
\end{haiku}

\begin{haiku}
    {\FH 二もとの梅に遅速を愛すかな}\hfill{\FH 同}

    {\FK 迟速各有致心爱两株梅}
\end{haiku}

\begin{haiku}
    {\FH 真\ruby{金}{がね}\ruby{食}{は}む鼠の\ruby{牙}{きば}の音さむし}\hfill{\FH 同}

    {\FK 饥鼠食坚铁铮铮牙音寒}
\end{haiku}

\begin{haiku}
    {\FH 手燭して色\ruby{失}{うしな}へる黄菊かな}\hfill{\FH 同}

    {\FK 黄菊失颜色烛台蜡炬明}
\end{haiku}

\begin{haiku}
    {\FH 蘆の花\ruby{漁翁}{むらぎみ}が宿の煙飛ぶ}\hfill{\FH 同}

    {\FK 渔家炊烟起汀洲芦花深}
\end{haiku}

\begin{haiku}
    {\FH 三径の十歩に尽きて\ruby{蓼}{たで}の花}\hfill{\FH 同}

    {\FK 三径十步尽㛹娟蓼花中}
\end{haiku}

\begin{haiku}
    {\FH 葛の葉のうらみ顔なる\ruby{細雨}{さいう}かな}\hfill{\FH 同}

    {\FK 葛叶蒙细雨花容恨怎休}
\end{haiku}

\begin{haiku}
    {\FH 父が着てわが着て古りし秋袷}\hfill{\FH 同}

    {\FK 父子同衣此秋袷岁月多}
\end{haiku}

\begin{haiku}
    {\FH \ruby{耕}{たがやし}や鳥さへ啼かぬ山陰に}\hfill{\FH 同}

    {\FK 山阴躬耕处竟无野鸟啼}
\end{haiku}

\begin{haiku}
    {\FH 僧とめてうれしと蚊帳も高くつる}\hfill{\FH 同}

    {\FK 欢喜留僧住悬挂蚊帐高}
\end{haiku}

\begin{haiku}
    {\FH 涼しさや鐘をはなるる鐘のこえ}\hfill{\FH 同}

    {\FK 秋日生凉意金声离钟飞}
\end{haiku}

\begin{haiku}
    {\FH 行き行きてここに行き行く夏野かな}\hfill{\FH 同}

    {\FK 行行重行行行行复行行此处步夏日原野无穷尽}
\end{haiku}

\begin{haiku}
    {\FH 河豚の面世上の人を\ruby{白眼}{にらむ}哉}\hfill{\FH 同}

    {\FK 河豚翻白眼冷然观世人}
\end{haiku}

\begin{haiku}
    {\FH 秋風の動かして行く案山子かな}\hfill{\FH 同}

    {\FK 草人失怙恃动摇随秋风}
\end{haiku}

\begin{haiku}
    {\FH 笠とれて面目もなき案山子かな}\hfill{\FH 同}

    {\FK 萆人除斗笠面目俱已非}
\end{haiku}

\begin{haiku}
    {\FH 行春や白き花みゆ垣のひま}\hfill{\FH 同}

    {\FK 四月春归去垣隙见白花}
\end{haiku}

\begin{haiku}
    {\FH 不二\ruby{颪}{おろし}十三州の柳かな}\hfill{\FH 同}

    {\FK 不二山风下柳翠十三州}
\end{haiku}

\begin{haiku}
    {\FH 嵯峨へ帰る人はいづこの花に暮し}\hfill{\FH 同}

    {\FK 随处樱花暮行人返嵯峨}
\end{haiku}

\begin{haiku}
    {\FH 青梅に眉あつめたる美人かな}\hfill{\FH 同}

    {\FK 双弯蛾眉颦美人食青梅}
\end{haiku}

\begin{haiku}
    {\FH 月天心貧しき町を通りけり}\hfill{\FH 同}

    {\FK 行过穷巷里良月似天心}
\end{haiku}

\begin{haiku}
    {\FH 春雨やゆるい\ruby{下駄}{げた}借す奈良の宿}\hfill{\FH 同}

    {\FK 借得宽木屐春雨宿奈良}
\end{haiku}

\begin{haiku}
    {\FH 蛇の衣吹かれ出でたる\ruby{籬}{まがき}かな}\hfill{\FH 同}

    {\FK 随风飘篱外六月蛇蜕衣}
\end{haiku}

\begin{haiku}
    {\FH 仮寝する\ruby{暇}{いとま}を花の主かな}\hfill{\FH 同}

    {\FK 身为名花主得暇假寝中}
\end{haiku}

\begin{haiku}
    {\FH 水仙やさむき都の\ruby{此処}{ここ}\ruby{彼処}{かしこ}}\hfill{\FH 同}

    {\FK 京中寒威重水仙处处开}
\end{haiku}

\begin{haiku}
    {\FH 古井戸や蚊にとぶ魚の音くらし}\hfill{\FH 同}

    {\FK 古井音响暗跃鱼扑飞蚊}
\end{haiku}

\begin{haiku}
    {\FH 牡丹散りて打ち重なりぬ二三片}\hfill{\FH 同}

    {\FK 重叠两三片牡丹飞散中}
\end{haiku}

\begin{haiku}
    {\FH 行く春は重き琵琶の抱ぎこころ}\hfill{\FH 同}

    {\FK 琵琶觉沉重残春怀抱中}
\end{haiku}

\begin{haiku}
    {\FH 春の水山なき国を流れけり}\hfill{\FH 同}

    {\FK 此乡无冈峦阳春行水流}
\end{haiku}

\begin{haiku}
    {\FH 客僧の二階下り来る野分かな}\hfill{\FH 同}

    {\FK 凉秋烈风至客僧下二楼}
\end{haiku}

\begin{haiku}
    {\FH 野路の秋我が後ろより人や来る}\hfill{\FH 同}

    {\FK 人自身后至我行野路秋}
\end{haiku}

\begin{haiku}
    {\FH \ruby{寂}{せき}として客の\ruby{絶間}{たえま}の牡丹かな}\hfill{\FH 同}

    {\FK 宾客绝过往牢落牡丹花}
\end{haiku}

\begin{haiku}
    {\FH 寒梅や火の\ruby{迸}{ほとばし}る\ruby{鐵}{まがね}より}\hfill{\FH 同}

    {\FK 炽铁迸烈火寒梅色更红}
\end{haiku}

\begin{haiku}
    {\FH \ruby{石工}{いしきり}の\ruby{鑿}{のみ}\ruby{冷}{ひや}したる清水かな}\hfill{\FH 同}

    {\FK 石工斧凿冷炎夏泉水清}
\end{haiku}

\begin{haiku}
    {\FH 菊作り汝は菊の\ruby{奴}{やっこ}かな}\hfill{\FH 同}

    {\FK 寒菊细护理汝为黄花奴}
\end{haiku}

\begin{haiku}
    {\FH 川狩や帰去来という声すなり}\hfill{\FH 同}

    {\FK 捕鱼河川上声如归去来}
\end{haiku}

\begin{haiku}
    {\FH \ruby{鋸}{のこぎり}の音貧しさよ夜半の冬}\hfill{\FH 同}

    {\FK 隆冬值夜半锯音识贫寒}
\end{haiku}

\begin{haiku}
    {\FH 小鳥来る音うれしさよ\ruby{板庇}{いたびさし}}\hfill{\FH 同}

    {\FK 清音诚可喜小鸟来板窗}
\end{haiku}

\begin{haiku}
    {\FH 玉人の\ruby{座右}{ざゆう}に開く椿かな}\hfill{\FH 同}

    {\FK 山茶正吐艳玉人座右开}
\end{haiku}

\begin{haiku}
    {\FH 雨にとまる玉水の宿の蝸牛}\hfill{\FH 同}

    {\FK 雨水频滴落蜗牛宿积潦}
\end{haiku}

\begin{haiku}
    {\FH 滝口に灯を呼ぶ声や春の雨}\hfill{\FH 同}

    {\FK 瀑口呼灯火闻声春雨中}
\end{haiku}

\begin{haiku}
    {\FH 五月雨や\ruby{滄海}{あおうみ}を\ruby{衝}{つ}く\ruby{濁水}{にごりみず}}\hfill{\FH 同}

    {\FK 浊水冲沧海五月梅雨赊}
\end{haiku}

\begin{haiku}
    {\FH 枕上秋の夜をまもる刀かな}\hfill{\FH 同}

    {\FK 霜锋守秋夜枕上撗宝刀}
\end{haiku}

\begin{haiku}
    {\FH 水際もなくて古江の時雨かな}\hfill{\FH 同}

    {\FK 辽阔无边际时雨洒古江}
\end{haiku}

\begin{haiku}
    {\FH 山を越す人に別れてかれ野かな}\hfill{\FH 同}

    {\FK 行人越山去枯野话分携}
\end{haiku}

\begin{haiku}
    {\FH 斧入れて香におどろくや冬木立}\hfill{\FH 同}

    {\FK 冬木萧条立斧入惊树香}
\end{haiku}

\begin{haiku}
    {\FH \ruby{半江}{はんこう}の\ruby{斜日片雲}{しゃじつへんうん}の時雨かな}\hfill{\FH 同}

    {\FK 片云挟时雨斜日照半江}
\end{haiku}

\begin{haiku}
    {\FH しぐるるや我も古人の夜に似たる}\hfill{\FH 同}

    {\FK 今宵时雨降心情似古人}
\end{haiku}

\begin{haiku}
    {\FH 父母の事の見おもう秋の暮}\hfill{\FH 同}

    {\FK 只是念父母秋暮怀想中}
\end{haiku}

\begin{haiku}
    {\FH 大雪となりけり関のとざし時}\hfill{\FH 同}

    {\FK 满空大雪舞寒重且闭关}
\end{haiku}

\begin{haiku}
    {\FH 手燭して庭ふむ人や春惜しむ}\hfill{\FH 同}

    {\FK 持烛行庭内人惜春光归}
\end{haiku}

\begin{haiku}
    {\FH しののめに小雨降出す焼野かな}\hfill{\FH 同}

    {\FK 原野烈火后拂晓细雨滋}
\end{haiku}

\begin{haiku}
    {\FH おちこちおちこちとうつ砧かな}\hfill{\FH 同}

    {\FK 处处清夜砧远近成秋声}
\end{haiku}

\begin{haiku}
    {\FH \ruby{短}{みじか}夜や枕に近き銀屛風}\hfill{\FH 同}

    {\FK 短夜须臾过近枕银屏风}
\end{haiku}

\begin{haiku}
    {\FH うき我に砧うて今は又やみね}\hfill{\FH 同}

    {\FK 砧声起又止寒声袭愁人}
\end{haiku}

\begin{haiku}
    {\FH 凩や\ruby{碑}{いしぶみ}をよむ僧一人}\hfill{\FH 同}

    {\FK 凉秋金风起一僧读碑文}
\end{haiku}

\begin{haiku}
    {\FH 甲斐が嶺に雲こそ\ruby{懸}{かか}れ梨の花}\hfill{\FH 同}

    {\FK 云悬甲斐岭梨花晚春开}
\end{haiku}

\begin{haiku}
    {\FH 帛を裂く琵琶の流れや秋の声}\hfill{\FH 同}

    {\FK 琵琶如裂帛秋声乃尔流}
\end{haiku}

\begin{haiku}
    {\FH 秋雨や水底の草を踏み渡る}\hfill{\FH 同}

    {\FK 足践水底草秋雨渡河来}
\end{haiku}

\begin{haiku}
    {\FH \ruby{野袴}{のばかま}の法師が旅や春の風}\hfill{\FH 同}

    {\FK 客途春风里法师着宽裙}
\end{haiku}

\begin{haiku}
    {\FH 日かえりの\ruby{兀}{はげ}山越ゆる暑さかな}\hfill{\FH 同}

    {\FK 日落天犹暑人行过兀山}
\end{haiku}

\begin{haiku}
    {\FH 海手より日は照つけて山さくら}\hfill{\FH 同}

    {\FK 海边来日照山头樱花开}
\end{haiku}

\begin{haiku}
    {\FH \ruby{居}{すわ}りたる舟に寝てゐる暑さかな}\hfill{\FH 同}

    {\FK 炎暑唯解卧人在舟船中}
\end{haiku}

\begin{haiku}
    {\FH 白蓮を剪らんとぞ思う僧のさま}\hfill{\FH 同}

    {\FK 欲剪白莲否法师何所思}
\end{haiku}

\begin{haiku}
    {\FH 広庭の牡丹や天の一方に}\hfill{\FH 同}

    {\FK 广庭牡丹放今在天一方}
\end{haiku}

\begin{haiku}
    {\FH 蛇を\ruby{截}{きっ}てわたる谷路の若葉かな}\hfill{\FH 同}

    {\FK 截断游蛇路谷间嫩叶多}
\end{haiku}

\begin{haiku}
    {\FH 春の\ruby{夕}{くれ}たえなむとする香をつく}\hfill{\FH 同}

    {\FK 春夕无限好阵阵芳馨来}
\end{haiku}

\begin{haiku}
    {\FH 狐火や髑髏に雨のたまる夜に}\hfill{\FH 同}

    {\FK 入夜磷火现髑髅承雨多}
\end{haiku}

\begin{haiku}
    {\FH 夜水とる里人のこえや夏の月}\hfill{\FH 同}

    {\FK 夏夜月色里乡人汲水声}
\end{haiku}

\begin{haiku}
    {\FH 春雨や蛙の腹はまだぬれず}\hfill{\FH 同}

    {\FK 蛙腹犹未湿一任春雨多}
\end{haiku}

\begin{haiku}
    {\FH 夕立や筆も乾かず一千言}\hfill{\FH 同}

    {\FK 夕暮白雨至千言笔不干}
\end{haiku}

\begin{haiku}
    {\FH 時雨音なくて苔に昔を\ruby{偲}{しの}ぶかな}\hfill{\FH 同}

    {\FK 芭蕉忌辰}

    {\FK 前尘影事成追忆时雨无声滴苍苔}
\end{haiku}

\begin{haiku}
    {\FH 温泉の底にわが足見ゆる今朝の秋}\hfill{\FH 同}

    {\FK 俯视吾足温泉底立秋飒爽在今朝}
\end{haiku}

\begin{haiku}
    {\FH 三井寺や日は午にせまる若楓}\hfill{\FH 同}

    {\FK 日到中天逼午近三井寺里有嫩枫}
\end{haiku}

\begin{haiku}
    {\FH 鳥羽殿へ五六騎いそぐ野分かな}\hfill{\FH 同}

    {\FK 驰驱前途鸟羽殿五六骑行野风中}
\end{haiku}

\begin{haiku}
    {\FH 匂いある衣も畳まず春の暮}\hfill{\FH 同}

    {\FK 熏香衣衫未整叠春日和煦向晚时}
\end{haiku}

\begin{haiku}
    {\FH 秋風や酒肆にうたう漁者樵者}\hfill{\FH 同}

    {\FK 秋风瑟瑟过酒肆渔父樵人共唱酬}
\end{haiku}

\begin{haiku}
    {\FH 春の海終日のたりのたりかな}\hfill{\FH 同}

    {\FK 终日悠悠是春海波光潋艳浪花柔}
\end{haiku}

\begin{haiku}
    {\FH 夕風や水青鷺の脛をうつ}\hfill{\FH 同}

    {\FK 夕风吹水漪涟行飞沫溅湿青鹭胫}
\end{haiku}

\begin{haiku}
    {\FH 夏河を越すうれしさよ手に草履}\hfill{\FH 同}

    {\FK 夏日此事足娱心手持芒鞋涉川来}
\end{haiku}

\begin{haiku}
    {\FH 梅遠近南すべくや北すべく}\hfill{\FH 同}

    {\FK 寒梅飘香颜色丽远近南北处处开}
\end{haiku}

\begin{haiku}
    {\FH 半日の閑を榎や蝉の声}\hfill{\FH 同}

    {\FK 静听蝉鸣朴树上又得浮生半日闲}
\end{haiku}

\begin{haiku}
    {\FH 夕立や草葉をつかむ群雀}\hfill{\FH 同}

    {\FK 夕暮时分来风雨群雀乱聚草叶中}
\end{haiku}

\begin{haiku}
    {\FH 秋立つや\ruby{白湯}{さゆ}\ruby{香}{かうば}しき\ruby{施薬}{せやく}院}\hfill{\FH 同}

    {\FK 立秋飒爽今日至施药院中白汤香}
\end{haiku}

\begin{haiku}
    {\FH 霜百里舟中に我月を領す}\hfill{\FH 同}

    {\FK 身在舟中月属我两岸清霜百里溥}
\end{haiku}

\begin{haiku}
    {\FH 花を踏し草履も見えて朝寝かな}\hfill{\FH 同}

    {\FK 朝晨高卧人未起草履曾是踏花来}
\end{haiku}

\begin{haiku}
    {\FH 夕かすみおもえば\ruby{隔}{へだ}つ昔かな}\hfill{\FH 几董}

    {\FK 晋子七十年怀古}

    {\FK 往昔成隔世沉吟对夕霞}
\end{haiku}

\begin{haiku}
    {\FH 水仙にたまる\ruby{師走}{しはす}の\ruby{埃}{ほこり}かな}\hfill{\FH 同}

    {\FK 冬日庭院静水仙蒙尘埃}
\end{haiku}

\begin{haiku}
    {\FH 朧夜や南下りにひがしやま}\hfill{\FH 同}

    {\FK 南下东山去朦胧春夜中}
\end{haiku}

\begin{haiku}
    {\FH 蠅うっていささか汚す団扇かな}\hfill{\FH 同}

    {\FK 团扇扑蝇后遂教微玷污}
\end{haiku}

\begin{haiku}
    {\FH 大雪のものしずかさや明の春}\hfill{\FH 同}

    {\FK 大雪人世静明朝新春来}
\end{haiku}

\begin{haiku}
    {\FH \ruby{埒}{らち}もなき荊が中の野梅かな}\hfill{\FH 同}

    {\FK 幽径}

    {\FK 四周无围垣野梅荆棘中}
\end{haiku}

\begin{haiku}
    {\FH 大佛をみかけて遠き冬野かな}\hfill{\FH 同}

    {\FK 大佛遥可见冬野远迢迢}
\end{haiku}

\begin{haiku}
    {\FH 磯山や小松が中を春の水}\hfill{\FH 同}

    {\FK 彳亍小松里矶山春水流}
\end{haiku}

\begin{haiku}
    {\FH しぐるるや南に\ruby{低}{ひく}き雲の峰}\hfill{\FH 同}

    {\FK 时雨飘然降云峰向南低}
\end{haiku}

\begin{haiku}
    {\FH 春雨や簑の下なる恋衣}\hfill{\FH 同}

    {\FK 少年行}

    {\FK 蓑下恋衣现膏润春雨中}
\end{haiku}

\begin{haiku}
    {\FH あるじなき几帳にとまる蛍かな}\hfill{\FH 同}

    {\FK 夕殿萤飞思悄然}

    {\FK 几帐今无主唯有流萤栖}
\end{haiku}

\begin{haiku}
    {\FH 行く舟に遠近かわる砧かな}\hfill{\FH 同}

    {\FK 砧声变远近一路送行舟}
\end{haiku}

\begin{haiku}
    {\FH 旅人にわが\ruby{糧}{かて}わかつ深雪かな}\hfill{\FH 同}

    {\FK 共困深雪里分粮赠旅人}
\end{haiku}

\begin{haiku}
    {\FH 恋々として柳遠のく舟路かな}\hfill{\FH 同}

    {\FK 恋恋柳远去客舟征途遥}
\end{haiku}

\begin{haiku}
    {\FH 浮く魚の影は底行く清水かな}\hfill{\FH 同}

    {\FK 影行清水底游鱼随波浮}
\end{haiku}

\begin{haiku}
    {\FH 短夜や空をわかるる海の色}\hfill{\FH 同}

    {\FK 冬夜转瞬过海色长天分}
\end{haiku}

\begin{haiku}
    {\FH 春くれぬ酔中の詩に墨ぬらん}\hfill{\FH 同}

    {\FK 思将墨涂却暮春曾醉吟}
\end{haiku}

\begin{haiku}
    {\FH 草の戸や秋の日落ちて秋の月}\hfill{\FH 同}

    {\FK 草庵秋日落又见秋月明}
\end{haiku}

\begin{haiku}
    {\FH 冬木立月\ruby{骨髄}{こつずい}に入る夜かな}\hfill{\FH 同}

    {\FK 夜月沁透骨髄里枯木凄凉立隆冬}
\end{haiku}

\begin{haiku}
    {\FH 十六夜や一人\ruby{缺}{か}けたる月の友}\hfill{\FH 同}

    {\FK 悼太祇}

    {\FK 十六夜间添惆怅赏月良友少一人}
\end{haiku}

\begin{haiku}
    {\FH 来る雁に\ruby{儚}{はかな}き事を聞く夜かな}\hfill{\FH 同}

    {\FK 悼千代尼}

    {\FK 今夜鸿雁传噩耗始觉无常是浮生}
\end{haiku}

\begin{haiku}
    {\FH 父が酔家の新酒のうれしさに}\hfill{\FH 召波}

    {\FK 家酿新酒美老父赴醉乡}
\end{haiku}

\begin{haiku}
    {\FH 月\ruby{更}{ふ}けて桑に音ある\ruby{蚕}{かいこ}かな}\hfill{\FH 同}

    {\FK 夜月渐沉沉桑中闻蚕音}
\end{haiku}

\begin{haiku}
    {\FH 鶏合左右百羽を分ちけり}\hfill{\FH 同}

    {\FK 群鸡正斗争左右百羽分}
\end{haiku}

\begin{haiku}
    {\FH 羊煮て兵を\ruby{労}{ねぎら}う霜夜かな}\hfill{\FH 同}

    {\FK 严霜乘夜降煮羊劳干城}
\end{haiku}

\begin{haiku}
    {\FH しづかさや雨の後なる春の水}\hfill{\FH 同}

    {\FK 自从雨过后春水悄无声}
\end{haiku}

\begin{haiku}
    {\FH 沖にふる小雨に入るや春の雁}\hfill{\FH 同}

    {\FK 海上小雨降春雁入空濛}
\end{haiku}

\begin{haiku}
    {\FH 無人境鶯庭をありきけり}\hfill{\FH 同}

    {\FK 来步无人境黄莺下庭中}
\end{haiku}

\begin{haiku}
    {\FH \ruby{松明}{たいまつ}に露の白さや夜の道}\hfill{\FH 同}

    {\FK 照映清露白夜行燃松明}
\end{haiku}

\begin{haiku}
    {\FH 怪談の後ろ更け行く夜寒かな}\hfill{\FH 同}

    {\FK 畅谈怪异后骎骎夜更寒}
\end{haiku}

\begin{haiku}
    {\FH 長き夜の寝さめ語るや父と母}\hfill{\FH 同}

    {\FK 父母共闲话寝觉长夜间}
\end{haiku}

\begin{haiku}
    {\FH 古き戸に影うつり行燕かな}\hfill{\FH 同}

    {\FK 影映古户牖飞燕去悠悠}
\end{haiku}

\begin{haiku}
    {\FH しられども秋の身に添う\ruby{金気}{かなけ}かな}\hfill{\FH 同}

    {\FK 蒙蒙犹未察金气入我身}
\end{haiku}

\begin{haiku}
    {\FH 何を釣沖の小舟ぞ笠の雪}\hfill{\FH 同}

    {\FK 笠上被雪钓何物小舟出没海波间}
\end{haiku}

\begin{haiku}
    {\FH 口切りや寺へ呼ばれて竹の奥}\hfill{\FH 同}

    {\FK 人在竹林幽深处呼入寺中烹新茶}
\end{haiku}

\begin{haiku}
    {\FH 秋惜しと一声虫のなく音かな}\hfill{\FH 大魯}

    {\FK 似是惜秋尽寒虫鸣一声}
\end{haiku}

\begin{haiku}
    {\FH うしろより雨の追来る焼野かな}\hfill{\FH 同}

    {\FK 人行烧野里身后雨追来}

\end{haiku}

\begin{haiku}
    {\FH 牡丹折りし父の\ruby{怒}{}ぞなつかしき}\hfill{\FH 同}

    {\FK 怀旧}

    {\FK 追忆严父怒因我折牡丹}
\end{haiku}

\begin{haiku}
    {\FH 夢破れいかり\ruby{衾}{ふすま}破れて君みえず}\hfill{\FH 同}

    {\FK 梦破衾亦裂无由得见君}
\end{haiku}

\begin{haiku}
    {\FH 濱\ruby{荻}{おぎ}のこえや潮に濡れながら}\hfill{\FH 月居}

    {\FK 滨荻沙沙动潮来湿此声}
\end{haiku}

\begin{haiku}
    {\FH たそがれの重き草履や桜人}\hfill{\FH 百池}

    {\FK 黄昏草履重赏樱道路长}
\end{haiku}

\begin{haiku}
    {\FH 照る月や京のぐるりは竹の雪}\hfill{\FH 同}

    {\FK 夜月倾流照竹雪围京华}
\end{haiku}

\begin{haiku}
    {\FH 雲起て寺門を出づる秋の声}\hfill{\FH 暁台}

    {\FK 云起出寺门萧条闻秋声}
\end{haiku}

\begin{haiku}
    {\FH 冬の情月明らかにあられ降る}\hfill{\FH 同}

    {\FK 月明素霰落寒冬亦有情}
\end{haiku}

\begin{haiku}
    {\FH 緑深く夕雨めぐる嵐山}\hfill{\FH 同}

    {\FK 夕雨来环绕岚山翠意深}
\end{haiku}

\begin{haiku}
    {\FH 蠅一つ我をめぐるや冬籠}\hfill{\FH 同}

    {\FK 隆冬闭门坐一蝇绕我飞}
\end{haiku}

\begin{haiku}
    {\FH 暁や鯨の\ruby{吼}{ほ}ゆる霜の海}\hfill{\FH 同}

    {\FK 拂晓长鲸吼海上寒霜浓}
\end{haiku}

\begin{haiku}
    {\FH 秋の山ところどころに煙立つ}\hfill{\FH 同}

    {\FK 凉秋临峰峦炊烟处处浮}
\end{haiku}

\begin{haiku}
    {\FH 風かなし夜々に缺けゆく月の形}\hfill{\FH 同}

    {\FK 秋风凄然过盈月夜夜亏}
\end{haiku}

\begin{haiku}
    {\FH 夜はうれしく昼は静なり春の雨}\hfill{\FH 樗良}

    {\FK 昼静夜可喜好雨春发生}
\end{haiku}

\begin{haiku}
    {\FH ふるさとへ戻りて見たる柳かな}\hfill{\FH 同}

    {\FK 依依见翠柳行子返故乡}
\end{haiku}

\begin{haiku}
    {\FH \ruby{別}{あか}れ寒し少し送ればまた寒し}\hfill{\FH 同}

    {\FK 别时已觉冷稍送更增寒}
\end{haiku}

\begin{haiku}
    {\FH 白梅や垣の内外にこぼれちる}\hfill{\FH 同}

    {\FK 墙垣内外满白梅今落花}
\end{haiku}

\begin{haiku}
    {\FH 山寺や誰も参らぬ涅槃像}\hfill{\FH 同}

    {\FK 山寺无人到寂寞涅槃图}
\end{haiku}

\begin{haiku}
    {\FH 雲高く風たえて花の嵐山}\hfill{\FH 同}

    {\FK 云高风断绝樱花满岚山}
\end{haiku}

\begin{haiku}
    {\FH 鳥の羽に\ruby{見初}{みそむ}る春の光かな}\hfill{\FH 同}

    {\FK 春来初领略鸟羽生光华}
\end{haiku}

\begin{haiku}
    {\FH 袖口に日の色うれし今朝の春}\hfill{\FH 同}

    {\FK 袖口日影乐立春在今朝}
\end{haiku}

\begin{haiku}
    {\FH 涼しさや松の木の間より\ruby{洩}{も}るる風}\hfill{\FH 同}

    {\FK 由此觉凉意风从松间来}
\end{haiku}

\begin{haiku}
    {\FH 花に\ruby{狂}{くる}ひ月に驚く胡蝶かな}\hfill{\FH 同}

    {\FK 风流属蝴蝶惊月为花狂}
\end{haiku}

\begin{haiku}
    {\FH 秋はうし古里へ帰るわかれさえ}\hfill{\FH 同}

    {\FK 凉秋别离多惆怅纵令此去返故乡}
\end{haiku}

\begin{haiku}
    {\FH \ruby{冷}{ひや}水に\ruby{煎餅}{せんべい}二枚樗良が夏}\hfill{\FH 同}

    {\FK 自画赞}

    {\FK 煎饼二枚佐冷水樗良夏日如此消}
\end{haiku}

\begin{haiku}
    {\FH \ruby{伐}{きり}倒す楠匂ひけりくれの春}\hfill{\FH 闌更}

    {\FK 楠木今伐倒春尽送芬芳}
\end{haiku}

\begin{haiku}
    {\FH 亡き迹や針も錆びゆく秋の雨}\hfill{\FH 同}

    {\FK 楚江丧妻}

    {\FK 人亡针亦锈凄凉秋雨中}
\end{haiku}

\begin{haiku}
    {\FH 瀬がしらや光とびつつ夏の月}\hfill{\FH 同}

    {\FK 流水此处疾夏夜月飞光}
\end{haiku}

\begin{haiku}
    {\FH \ruby{鞘}{さや}あかき長刀行くや春の野辺}\hfill{\FH 同}

    {\FK 春日行原野长刀贮赤鞘}
\end{haiku}

\begin{haiku}
    {\FH 桜さき桜散りつつ我老いぬ}\hfill{\FH 同}

    {\FK 我于此时老樱花落又开}
\end{haiku}

\begin{haiku}
    {\FH 春風や草木に動く日の光}\hfill{\FH 同}

    {\FK 动摇草木上春风日光明}
\end{haiku}

\begin{haiku}
    {\FH 桐一葉裏も表も青かりし}\hfill{\FH 蒼虬}

    {\FK 梧桐一叶落表里何青青}
\end{haiku}

\begin{haiku}
    {\FH 行きゆけば左右になるや灯と砧}\hfill{\FH 梅室}

    {\FK 人行顾左右灯影与砧声}
\end{haiku}

\begin{haiku}
    {\FH 天の川星より上にみゆるかな}\hfill{\FH 白雄}

    {\FK 更在繁星上迢迢银汉高}
\end{haiku}

\begin{haiku}
    {\FH こがらしや潮ながら飛ぶ浜の砂}\hfill{\FH 同}

    {\FK 凄厉寒风至潮生滨砂飞}
\end{haiku}

\begin{haiku}
    {\FH 山そびえ川ながれたり秋の風}\hfill{\FH 蓼太}

    {\FK 萧然秋风至山耸大江流}
\end{haiku}

\begin{haiku}
    {\FH 鹿の音の嵯峨へ下りたる夜寒かな}\hfill{\FH 同}

    {\FK 遂觉夜寒重鹿音下嵯峨}
\end{haiku}

\begin{haiku}
    {\FH 我が影の壁にしむ夜やきりぎりす}\hfill{\FH 同}

    {\FK 我影凝壁上夜闻寒蛩声}
\end{haiku}

\begin{haiku}
    {\FH ここかしこまだ寝ぬ里の蚊遣かな}\hfill{\FH 同}

    {\FK 里人犹未寝处处燃蚊香}
\end{haiku}

\begin{haiku}
    {\FH 世の中は三日見ぬ間に桜かな}\hfill{\FH 同}

    {\FK 时隔才三日人世满樱花}
\end{haiku}

\begin{haiku}
    {\FH 鹿もよく寝て朧なり奈良の月}\hfill{\FH 同}

    {\FK 麋鹿亦沉睡奈良月朦胧}
\end{haiku}

\begin{haiku}
    {\FH \ruby{曙}{あけぼの}の青き中より一葉かな}\hfill{\FH 同}

    {\FK 一叶分明现青翠曙色中}
\end{haiku}

\begin{haiku}
    {\FH 白雪の中に灯ともす野守かな}\hfill{\FH 同}

    {\FK 值夜人守原野里白雪中有一灯明}
\end{haiku}

\begin{haiku}
    {\FH 白萩や露一升に花一升}\hfill{\FH 同}

    {\FK 凉秋九月白萩放一升露水一升花}
\end{haiku}

\begin{haiku}
    {\FH 五月雨やある夜\ruby{密}{ひそ}かに松の月}\hfill{\FH 同}

    {\FK 一夜松际露微月五月浸霪梅雨长}
\end{haiku}

\begin{haiku}
    {\FH 句を煉て腸うごく霜夜がな}\hfill{\FH 太祇}

    {\FK 炼句枯肠动霜夜费思量}
\end{haiku}

\begin{haiku}
    {\FH やぶ入りや琴かき鳴す親の前}\hfill{\FH 同}

    {\FK 游子归省去鸣琴亲长前}
\end{haiku}

\begin{haiku}
    {\FH 人追うて蜂もどりけり花の上}\hfill{\FH 同}

    {\FK 追人蜂又至还向花上栖}
\end{haiku}

\begin{haiku}
    {\FH 目をあいて聞いて居るなり四方の春}\hfill{\FH 同}

    {\FK 耳闻并目睹四方皆春回}
\end{haiku}

\begin{haiku}
    {\FH \ruby{東風}{こち}吹くと語りもぞ行く主と\ruby{従者}{ずさ}}\hfill{\FH 同}

    {\FK 主仆行共语东风今吹来}
\end{haiku}

\begin{haiku}
    {\FH 秋さびしおぼえたる句を皆申す}\hfill{\FH 同}

    {\FK 追忆旧句讽咏尽秋来无计遣寂寥}
\end{haiku}

\begin{haiku}
    {\FH 山吹や葉に花に葉に花に葉に}\hfill{\FH 同}

    {\FK 旖旎棠棣重重发花凌叶兮叶载花}
\end{haiku}

\begin{haiku}
    {\FH 寝よといふ寝覚めの\ruby{夫}{つま}や小夜砧}\hfill{\FH 同}

    {\FK 良人觉来嘱就寝清秋夜半闻砧声}
\end{haiku}

\begin{haiku}
    {\FH 初恋や燈籠によする顔と顔}\hfill{\FH 同}

    {\FK 共倚灯笼颊相偎初恋最是情深时}
\end{haiku}

\begin{haiku}
    {\FH 川中や獺なき歩行く朧月}\hfill{\FH 五明}

    {\FK 朦胧夜月下啼獭步中流}
\end{haiku}

\begin{haiku}
    {\FH 笛吹いて蝶寄する人\ruby{奇}{き}なるかな}\hfill{\FH 同}

    {\FK 奇哉粉蝶至来旁吹笛人}
\end{haiku}

\begin{haiku}
    {\FH 暴風暗く鷗みだれてこえかなし}\hfill{\FH 同}

    {\FK 鸥乱暴风暗鸣声良可哀}
\end{haiku}

\begin{haiku}
    {\FH 蠟燭をともして年を惜みけり}\hfill{\FH 同}

    {\FK 惋惜一年尽燃点蜡炬明}
\end{haiku}

\begin{haiku}
    {\FH 梅と我と年同じくて我老いたり}\hfill{\FH 同}

    {\FK 梅我年相若老矣叹今吾}
\end{haiku}

\begin{haiku}
    {\FH 雪二日積りて物の音もなし}\hfill{\FH 同}

    {\FK 积雪经二日物音渺不闻}
\end{haiku}

\begin{haiku}
    {\FH 仰に寝て銀河に胸を冷さばや}\hfill{\FH 同}

    {\FK 夏夜苦热乃仰卧思赴银河冷吾胸}
\end{haiku}

\begin{haiku}
    {\FH 見下せば人里ふかし山さくら}\hfill{\FH 麦水}

    {\FK 山中樱花放俯瞰墟里深}
\end{haiku}

\begin{haiku}
    {\FH 椿落ちて一僧笑ひ過ぎ行きぬ}\hfill{\FH 同}

    {\FK 一僧含笑过山茶落纷纷}
\end{haiku}

\begin{haiku}
    {\FH 山につき山にはなれつ秋の雲}\hfill{\FH 同}

    {\FK 触山又离去无定叹秋云}
\end{haiku}

\begin{haiku}
    {\FH 我寺の鐘とおもわず夕かすみ}\hfill{\FH 蝶夢}

    {\FK 钟声非我寺夕暮起晚霞}
\end{haiku}

\begin{haiku}
    {\FH うぐいすのかくれ現われ見えずなりぬ}\hfill{\FH 成美}

    {\FK 黄莺隐或现芳姿欲睹难}
\end{haiku}

\begin{haiku}
    {\FH 春の夢さめて隣のはなしかな}\hfill{\FH 同}

    {\FK 春梦今觉醒言语发邻家}
\end{haiku}

\begin{haiku}
    {\FH 君がため春の野に出でて若菜\ruby{摘}{つ}む}\hfill{\FH 抱一}

    {\FK 揽袖摘若菜情深总为君}
\end{haiku}

\begin{haiku}
    {\FH 峰となり渚となるや秋の雲}\hfill{\FH 同}

    {\FK 为岭为洲渚秋云形自多}
\end{haiku}

\begin{haiku}
    {\FH 秋風や舟より舟へ行く烏}\hfill{\FH 士朗}

    {\FK 鸦飞秋风里由此向彼舟}
\end{haiku}

\begin{haiku}
    {\FH 嬉しさにいくらも\ruby{解}{ほど}く\ruby{千巻}{ちまき}かな}\hfill{\FH 同}

    {\FK 欢欣食角粽绳结解几多}
\end{haiku}

\begin{haiku}
    {\FH 鶯に漕ぎはなれたる小船かな}\hfill{\FH 同}

    {\FK 悠然离岸去小舟莺声中}
\end{haiku}

\begin{haiku}
    {\FH ゆくとしのごそりともせぬ山家かな}\hfill{\FH 同}

    {\FK 一岁行看尽山家寂无声}
\end{haiku}

\begin{haiku}
    {\FH 誰\ruby{訪}{とぶら}ひて留守の釣瓶に杜若}\hfill{\FH 宮紫暁}

    {\FK 未悉主人访谁去杜若留守旁钓瓶}
\end{haiku}

\begin{haiku}
    {\FH 雨の芙蓉雪もつほどの\ruby{撓}{たわ}みかな}\hfill{\FH 紀梅亭}

    {\FK 芙蓉经雨如载雪遂教花萼皆低弯}
\end{haiku}

\begin{haiku}
    {\FH 山の井にあすまで残れ夏の月}\hfill{\FH 乙二}

    {\FK 夏月映山井残留至明朝}
\end{haiku}

\begin{haiku}
    {\FH 初夢にふるさとをみて涙かな}\hfill{\FH 一茶}

    {\FK 新岁初成梦客子返故乡天涯南柯觉清泪下数行}
\end{haiku}

\begin{haiku}
    {\FH 生残る我にかかるや草の露}\hfill{\FH 同}

    {\FK 悼父}

    {\FK 不肖悲后死草上凉露滋}
\end{haiku}

\begin{haiku}
    {\FH 秋の風親なき我を吹そぶり}\hfill{\FH 同}

    {\FK 其二}

    {\FK 父殁孤哀在一任秋风吹}
\end{haiku}

\begin{haiku}
    {\FH 我庵は草も夏痩したりけり}\hfill{\FH 同}

    {\FK 人微茅茨贱长夏草亦枯}
\end{haiku}

\begin{haiku}
    {\FH 今の世や猫も\ruby{杓子}{しゃく}も花見笠}\hfill{\FH 同}

    {\FK 世情知何似沐猴皆峨冠}
\end{haiku}

\begin{haiku}
    {\FH かたみ子や母がくるとて手をたたく}\hfill{\FH 同}

    {\FK 悼妻}

    {\FK 幼儿念慈躬拍手唤母来}
\end{haiku}

\begin{haiku}
    {\FH 此\ruby{便}{たより}聞とてある夜一時雨}\hfill{\FH 同}

    {\FK 悼同学中村桂国}

    {\FK 闻君从此去飒沓夜雨飞}
\end{haiku}

\begin{haiku}
    {\FH 我と来て遊べや親のない雀}\hfill{\FH 同}

    {\FK 孤雀毋心忧偕我共嬉游}
\end{haiku}

\begin{haiku}
    {\FH 有様は我も花より団子哉}\hfill{\FH 同}

    {\FK 箪瓢生尘久愿道心中情米团可充饥樱花不疗贫}
\end{haiku}

\begin{haiku}
    {\FH 極楽が近くなる身の寒さかな}\hfill{\FH 同}

    {\FK 渐与极乐近衰躯不耐寒}
\end{haiku}

\begin{haiku}
    {\FH 今の世は糸瓜に皮も売れにけり}\hfill{\FH 同}

    {\FK 瓜皮亦求市不古叹世风}
\end{haiku}

\begin{haiku}
    {\FH 秋の夜や目にふるるものみな遺品}\hfill{\FH 同}

    {\FK 悼妻}

    {\FK 满目皆遗物寂历秋夜中}
\end{haiku}

\begin{haiku}
    {\FH 一人と帳面につく夜寒かな}\hfill{\FH 同}

    {\FK 蚊帐拂吾面独宿觉夜寒}
\end{haiku}

\begin{haiku}
    {\FH 衣\ruby{替}{かえ}て坐って見てもひとりかな}\hfill{\FH 同}

    {\FK 更衣坐还看孑然一身零}
\end{haiku}

\begin{haiku}
    {\FH 福の来る門や野山の朝笑}\hfill{\FH 同}

    {\FK 福瑞来门闱野山朝开颜}
\end{haiku}

\begin{haiku}
    {\FH 老松や又あらためていく霞}\hfill{\FH 同}

    {\FK 栗之六十贺}

    {\FK 老松犹劲挺可见几许霞}
\end{haiku}

\begin{haiku}
    {\FH 秋風やむしりたがりし赤い花}\hfill{\FH 同}

    {\FK 悼女}

    {\FK 悼尔秋风里采集红花多}
\end{haiku}

\begin{haiku}
    {\FH \ruby{金釘}{かなくぎ}のやうな手足を秋の風}\hfill{\FH 同}

    {\FK 手足似铁钉沾疾卧秋风}
\end{haiku}

\begin{haiku}
    {\FH 年とりのあてもないぞよ旅烏}\hfill{\FH 同}

    {\FK 飘泊无定所羁旅又岁暮}
\end{haiku}

\begin{haiku}
    {\FH 月ちらり鶯ちらり夜は明ぬ}\hfill{\FH 同}

    {\FK 月微莺亦少行见曙色催}
\end{haiku}

\begin{haiku}
    {\FH 僧正の頭の上や蝿つるむ}\hfill{\FH 同}

    {\FK 飞蝇密密聚来集僧正头}
\end{haiku}

\begin{haiku}
    {\FH 我宿へ来さうにしたり\ruby{配}{くば}り餅}\hfill{\FH 同}

    {\FK 岁暮赠配饼何时来我门}
\end{haiku}

\begin{haiku}
    {\FH 春雨や猫におどりをおしえる子}\hfill{\FH 同}

    {\FK 小儿教猫舞春雨杜门时}
\end{haiku}

\begin{haiku}
    {\FH ゆさゆさとゆれいる東風の高木かな}\hfill{\FH 同}

    {\FK 东风撼高树枝叶尽动摇}
\end{haiku}

\begin{haiku}
    {\FH ゆさゆさと春が行ぞよのべの草}\hfill{\FH 同}

    {\FK 摇摇野原草春光今欲归}
\end{haiku}

\begin{haiku}
    {\FH \ruby{咬牙}{こうが}する人に目覚めて夜寒かな}\hfill{\FH 同}

    {\FK 寒夜梦惊觉人卧磨齿牙}
\end{haiku}

\begin{haiku}
    {\FH つくばうて一人とかなし冬の月}\hfill{\FH 同}

    {\FK 悼浪化上人}

    {\FK 悲凉独俯伏寒月清光中}
\end{haiku}

\begin{haiku}
    {\FH 夕暮や土とかたればちる木の葉}\hfill{\FH 同}

    {\FK 又悼桂国}

    {\FK 似与大地语夕暮木叶飞}
\end{haiku}

\begin{haiku}
    {\FH 身の秋や月は無きずの月ながら}\hfill{\FH 同}

    {\FK 悼妻}

    {\FK 良月纵无瑾寒来身知秋}
\end{haiku}

\begin{haiku}
    {\FH 初夢に猫も不二見る寝様かな}\hfill{\FH 同}

    {\FK 猫亦获初梦似见富士山}
\end{haiku}

\begin{haiku}
    {\FH \ruby{衰}{おとろ}えや花を折るにも口曲がる}\hfill{\FH 同}

    {\FK 残躯向衰暮折花口亦歪}
\end{haiku}

\begin{haiku}
    {\FH 手伝て虱を拾へ雀の子}\hfill{\FH 同}

    {\FK 啄取蚤虱起幼雀来相助}
\end{haiku}

\begin{haiku}
    {\FH 昼の蚊を後にかくす仏かな}\hfill{\FH 同}

    {\FK 昼蚊隐身后佛像箕踞中}
\end{haiku}

\begin{haiku}
    {\FH \ruby{蛬}{きりぎりす}三疋よれば喧嘩哉}\hfill{\FH 同}

    {\FK 三头蟾蜍聚定作喧哗声\footnote{\FT 原书把「蛬」误作「蟇」,蟋蟀变成了蟾蜍。}}
\end{haiku}

\begin{haiku}
    {\FH 人をさす草もへたへた枯にけり}\hfill{\FH 同}

    {\FK 可厌刺人草今见纷纷枯}
\end{haiku}

\begin{haiku}
    {\FH 赤い月是は誰のじゃ子ども達}\hfill{\FH 同}

    {\FK 儿童见赤月相问属谁家}
\end{haiku}

\begin{haiku}
    {\FH \ruby{貰}{もらう}よりはやくおとした\ruby{扇}{おうぎ}かな}\hfill{\FH 同}

    {\FK 乞得清风扇失却转瞬间}
\end{haiku}

\begin{haiku}
    {\FH 畠打の\ruby{真似}{まね}して歩く烏かな}\hfill{\FH 同}

    {\FK 乌鸦蹀躞行姿仿力田人}
\end{haiku}

\begin{haiku}
    {\FH 菊作りきくより白き\ruby{頭}{つぶり}哉}\hfill{\FH 同}

    {\FK 头比菊花白精心植寒英}
\end{haiku}

\begin{haiku}
    {\FH 蝸牛そろそろ登れ富士の山}\hfill{\FH 同}

    {\FK 自勉}

    {\FK 蜗牛徐徐行终登不二山}
\end{haiku}

\begin{haiku}
    {\FH 山焼の明りに下る夜舟かな}\hfill{\FH 同}

    {\FK 夜舟顺流下山头野火明}
\end{haiku}

\begin{haiku}
    {\FH 薮入や墓の松風うしろ吹}\hfill{\FH 同}

    {\FK 归省亲长墓身后来松风}
\end{haiku}

\begin{haiku}
    {\FH ほろつくや\ruby{誰}{たが}出代の涙雨}\hfill{\FH 同}

    {\FK 泣涕泪如雨驱遣是何人}
\end{haiku}

\begin{haiku}
    {\FH \ruby{悠然}{いうぜん}として山を見る\ruby{蛙}{かはづ}かな}\hfill{\FH 同}

    {\FK 此蛙安闲甚悠然望远山}
\end{haiku}

\begin{haiku}
    {\FH 葎からあんな小蝶が\ruby{生}{うま}れけり}\hfill{\FH 同}

    {\FK 蝴蝶如此艳竟出乱草中}
\end{haiku}

\begin{haiku}
    {\FH 昼の蚊の来るや手をかへ\ruby{品}{しな}をかへ}\hfill{\FH 同}

    {\FK 频频袭手足白昼蚊飞来}
\end{haiku}

\begin{haiku}
    {\FH \ruby{放}{はなれ}\ruby{鵜}{う}の子の鳴舟にもどりけり}\hfill{\FH 同}

    {\FK 幼鹈离亲鸟悲泣归舟中}
\end{haiku}

\begin{haiku}
    {\FH 涼風や力いっぱいきりぎりす}\hfill{\FH 同}

    {\FK 飒飒凉风过蟋蟀声弥强}
\end{haiku}

\begin{haiku}
    {\FH 湖水から出現したり雲の峰}\hfill{\FH 同}

    {\FK 似出湖水里云峰诚堪讶}
\end{haiku}

\begin{haiku}
    {\FH 木曽山に流入けり天の川}\hfill{\FH 同}

    {\FK 夜空银河壮流入木曾山}
\end{haiku}

\begin{haiku}
    {\FH 秋風や\ruby{磁石}{じしゃく}にあてる故郷山}\hfill{\FH 同}

    {\FK 磁石定方位秋风故乡山}
\end{haiku}

\begin{haiku}
    {\FH \ruby{孤}{みなしご}の我は光らぬ蛍かな}\hfill{\FH 同}

    {\FK 孤凄何似我流萤不生光}
\end{haiku}

\begin{haiku}
    {\FH 蚤の迹かぞへながらに添乳かな}\hfill{\FH 同}

    {\FK 蚤迹细寻觅婴儿哺乳中}
\end{haiku}

\begin{haiku}
    {\FH 魚どもは桶としらでや夕涼}\hfill{\FH 同}

    {\FK 未识处桶内夕凉鱼自游}
\end{haiku}

\begin{haiku}
    {\FH 散芒寒くなるのか目に見ゆる}\hfill{\FH 同}

    {\FK 严冬针芒散寒意入眼来}
\end{haiku}

\begin{haiku}
    {\FH 善尽し美を尽しても\ruby{芥子}{けし}の花}\hfill{\FH 同}

    {\FK 二十四年荣华只一夜梦}

    {\FK 尽善复尽美艳哉罂粟花}
\end{haiku}

\begin{haiku}
    {\FH 我庵や小川をかりて冷し瓜}\hfill{\FH 同}

    {\FK 借得小川水冷我庵中瓜}
\end{haiku}

\begin{haiku}
    {\FH おさな子や笑ふにつけて秋の暮}\hfill{\FH 同}

    {\FK 含笑对幼子凉秋日无多}
\end{haiku}

\begin{haiku}
    {\FH 家なしの此身も春に逢ふ日哉}\hfill{\FH 同}

    {\FK 元旦吟}

    {\FK 漂泊家何在今日又逢春}
\end{haiku}

\begin{haiku}
    {\FH 信濃路や山が荷になる暑さかな}\hfill{\FH 同}

    {\FK 炎暑信浓路山峦成重荷}
\end{haiku}

\begin{haiku}
    {\FH 蟻の道雲の峰よりつづきけん}\hfill{\FH 同}

    {\FK 长列蚁群起何处似从高天云峰来}
\end{haiku}

\begin{haiku}
    {\FH あばら骨なでじとすれど夜寒かな}\hfill{\FH 同}

    {\FK 手抚肋骨不自觉冬夜严寒来侵人}
\end{haiku}

\begin{haiku}
    {\FH 夕燕我には翌のあてもなし}\hfill{\FH 同}

    {\FK 明日蓬转定何处今夕偏见燕归来}
\end{haiku}

\begin{haiku}
    {\FH 露の世は露の世ながらさりながら}\hfill{\FH 同}

    {\FK 悼女}

    {\FK 浮生已与朝露同君行何复苦匆匆}
\end{haiku}

\begin{haiku}
    {\FH 名月をとってくれろと泣く子かな}\hfill{\FH 同}

    {\FK 欲取名月归掌上无知小儿长悲啼}
\end{haiku}

\begin{haiku}
    {\FH 世の中や鳴虫にさい上づ下手}\hfill{\FH 同}

    {\FK 世事如此何足论鸣虫亦欲分高低}
\end{haiku}

\begin{haiku}
    {\FH 門の月暑がへれば人もへる}\hfill{\FH 同}

    {\FK 纵然门前月长在减却暑热人亦稀}
\end{haiku}

\begin{haiku}
    {\FH 故郷は蠅まで人を刺しにけり}\hfill{\FH 同}

    {\FK 故乡于我情何薄摇翅苍蝇亦刺人}
\end{haiku}

\begin{haiku}
    {\FH わやわやと虫の上にも夜なべかな}\hfill{\FH 同}

    {\FK 入夜苦学人勤读任他唧唧虫声喧}
\end{haiku}

\begin{haiku}
    {\FH 近づきのらく書見へて秋の暮}\hfill{\FH 同}

    {\FK 熟谙乐谱今犹见惆怅凉秋欲尽时\footnote{\FT 此句译者将开头的日文「近づき」看错成「近すぎ」,原意是「好友」。后面的「楽書」望文生义作「乐谱」译,实为信手涂鸦之意。}}
\end{haiku}

\begin{haiku}
    {\FH 切木ともしらでや鳥の巣を作る}\hfill{\FH 同}

    {\FK 不知此木供樵采飞鸟犹来营居巢}
\end{haiku}

\begin{haiku}
    {\FH 横乗の馬のつゞくや夕雲雀}\hfill{\FH 同}

    {\FK 马队络绎人横坐夕阳消沉云雀声}
\end{haiku}

\begin{haiku}
    {\FH こんな夜は唐にもあろか時鳥}\hfill{\FH 同}

    {\FK 敢问唐土亦有否如此良夜时鸟鸣}
\end{haiku}

\begin{haiku}
    {\FH 身の上の鐘と知りつつ夕涼}\hfill{\FH 同}

    {\FK 他时撞钟或为我今日夕凉闻此声}
\end{haiku}

\begin{haiku}
    {\FH うつくしや\ruby{障子}{せうじ}の穴の天の川}\hfill{\FH 同}

    {\FK 病中}

    {\FK 长空银汉叹艳绝纸窗破穴仰望中}
\end{haiku}

\begin{haiku}
    {\FH これがまあ\ruby{終}{つひ}の\ruby{栖}{すみか}か雪五尺}\hfill{\FH 同}

    {\FK 生涯终此居庐否严冬雪降五尺深}
\end{haiku}

\begin{haiku}
    {\FH 月花や四十九年のむだ歩き}\hfill{\FH 同}

    {\FK 春花秋月何时了四十九年空蹒跚}
\end{haiku}

\begin{haiku}
    {\FH 痩蛙まけるな一茶是に有り}\hfill{\FH 同}

    {\FK 瘦蛙カ斗毋败北一茶在此与同仇}
\end{haiku}

\begin{haiku}
    {\FH あばら家や其身其まま明の春}\hfill{\FH 同}

    {\FK 故我此身还如旧陋栖明朝阳春来}
\end{haiku}

\begin{haiku}
    {\FH ふところに残るもの只梅の塵}\hfill{\FH 老鼠堂永機}

    {\FK 探怀何物在唯有梅尘留}
\end{haiku}

\begin{haiku}
    {\FH 武蔵野は皆鶯の栖かな}\hfill{\FH 春秋庵幹雄}

    {\FK 指点武藏野处处有莺栖}
\end{haiku}

\begin{haiku}
    {\FH 雨到る夏野の草や脛を没す}\hfill{\FH 其角堂永湖}

    {\FK 雨到夏野里丰草没胫深}
\end{haiku}

\begin{haiku}
    {\FH 夜あらしや今朝花も夢春も夢}\hfill{\FH 老鼠堂機一}

    {\FK 今朝花春俱是梦难堪昨夜风雨威}
\end{haiku}

\begin{haiku}
    {\FH 衰えや露の音にも幾寐覚}\hfill{\FH 同}

    {\FK 衰朽残年增叹惋几番梦觉因露溥}
\end{haiku}

\begin{haiku}
    {\FH 春風や山紫に水青し}\hfill{\FH 子規}

    {\FK 快哉春风至山紫水亦青}
\end{haiku}

\begin{haiku}
    {\FH 春風に零て赤し歯磨粉}\hfill{\FH 同}

    {\FK 牙粉染腥红点滴坠春风}
\end{haiku}

\begin{haiku}
    {\FH 聞き送る君が下駄遠き氷かな}\hfill{\FH 同}

    {\FK 声随君行远木屐履寒冰}
\end{haiku}

\begin{haiku}
    {\FH 城門を出て遠近の柳かな}\hfill{\FH 同}

    {\FK 出郭纵目望远近柳依依}
\end{haiku}

\begin{haiku}
    {\FH 隣からともしのうつる\ruby{芭蕉}{ばせを}哉}\hfill{\FH 同}

    {\FK 邻家灯影动闪烁映芭蕉}
\end{haiku}

\begin{haiku}
    {\FH 碁の音の林に響く夜寒かな}\hfill{\FH 同}

    {\FK 寒夜众响寂空林传棋声}
\end{haiku}

\begin{haiku}
    {\FH 目の前に顔のちらつく寒さかな}\hfill{\FH 同}

    {\FK 诣亡友墓}

    {\FK 音容宛然在萧寂觉幽寒}
\end{haiku}

\begin{haiku}
    {\FH 芙蓉花の或は恨み又は媚ぶ}\hfill{\FH 同}

    {\FK 或恨或骄态芙蓉姿诚多}
\end{haiku}

\begin{haiku}
    {\FH 遠山を二つに分けて日と時雨}\hfill{\FH 同}

    {\FK 晴雨自不齐远山两峰开}
\end{haiku}

\begin{haiku}
    {\FH 楓茂り桜茂りて寺くらし}\hfill{\FH 同}

    {\FK 遮护兰若暗郁郁枫樱茂}
\end{haiku}

\begin{haiku}
    {\FH 碁盤あり琴あり窓の竹の春}\hfill{\FH 同}

    {\FK 抚琴复敲枰竹窗春意饶}
\end{haiku}

\begin{haiku}
    {\FH 龍となり虎となる月の雲一片}\hfill{\FH 同}

    {\FK 如龙又似虎月照一片云}
\end{haiku}

\begin{haiku}
    {\FH 夏川や中流にしてかへり見る}\hfill{\FH 同}

    {\FK 中流且回首夏川荡客舟}
\end{haiku}

\begin{haiku}
    {\FH 五月雨の木曾は面白い処ぞや}\hfill{\FH 同}

    {\FK 送虚子赴木曾}

    {\FK 风光处处好梅雨洒木曾}
\end{haiku}

\begin{haiku}
    {\FH 人載せて牛載せて桃の渡しかな}\hfill{\FH 同}

    {\FK 载人载牛过暄阗桃花渡}
\end{haiku}

\begin{haiku}
    {\FH \ruby{韮}{かみら}剪って酒借りに行く隣かな}\hfill{\FH 同}

    {\FK 芳邻何碌碌剪韭借酒行}
\end{haiku}

\begin{haiku}
    {\FH 打ちやみつ打ちつ砧に恨あり}\hfill{\FH 同}

    {\FK 砧杵千万恨凭寄断续声}
\end{haiku}

\begin{haiku}
    {\FH 霞んだり曇ったり日の長さかな}\hfill{\FH 同}

    {\FK 日长风云幻霞生又朦胧}
\end{haiku}

\begin{haiku}
    {\FH 一村は杏と柳ばかりかな}\hfill{\FH 同}

    {\FK 一村唯杏柳乡里佳气多}
\end{haiku}

\begin{haiku}
    {\FH 胡蝶飛び風吹き胡蝶又来る}\hfill{\FH 同}

    {\FK 蝴蝶翩翩去随风更飞来}
\end{haiku}

\begin{haiku}
    {\FH 骨も見えずむくろも見えず草の花}\hfill{\FH 同}

    {\FK 吊古战场}

    {\FK 尸骨皆无睹唯见草花开}
\end{haiku}

\begin{haiku}
    {\FH 草の花人の死にしは昔なり}\hfill{\FH 同}

    {\FK 其二}

    {\FK 人死成终古荒陂草花开}
\end{haiku}

\begin{haiku}
    {\FH 水仙も處を得たり庭の隅}\hfill{\FH 同}

    {\FK 贺新居}

    {\FK 水仙亦得地庭隅自逍遥}
\end{haiku}

\begin{haiku}
    {\FH 若葉して都を下る隠士かな}\hfill{\FH 同}

    {\FK 送松宇先生}

    {\FK 隐士归何处落叶下京都}
\end{haiku}

\begin{haiku}
    {\FH いくたびも雪の深さを尋ねけり}\hfill{\FH 同}

    {\FK 寒空万玉舞数问雪浅深}
\end{haiku}

\begin{haiku}
    {\FH 聞きにゆけ須磨の隣の秋の風}\hfill{\FH 同}

    {\FK 送炼卿赴兵库}

    {\FK 须磨比邻处好去闻秋风}
\end{haiku}

\begin{haiku}
    {\FH 桃の如く\ruby{肥}{こ}えて可愛や目口鼻}\hfill{\FH 同}

    {\FK 千里女子照相}

    {\FK 可爱目口鼻人如桃实肥}
\end{haiku}

\begin{haiku}
    {\FH 金銀の色よ稻妻西東}\hfill{\FH 同}

    {\FK 横空金银色电光映西东}
\end{haiku}

\begin{haiku}
    {\FH 暑い日は思い出せよ富士の山}\hfill{\FH 同}

    {\FK 送友之英国}

    {\FK 若逢炎暑日还忆富士山}
\end{haiku}

\begin{haiku}
    {\FH 松青く梅白し誰が柴の戸ぞ}\hfill{\FH 同}

    {\FK 松青梅色白柴门是谁家}
\end{haiku}

\begin{haiku}
    {\FH 野分して蝉の少きあしたかな}\hfill{\FH 同}

    {\FK 今日秋风起明朝露蝉稀}
\end{haiku}

\begin{haiku}
    {\FH 長き夜の面白きかな水滸伝}\hfill{\FH 同}

    {\FK 披览水浒传长夜妙趣多}
\end{haiku}

\begin{haiku}
    {\FH あすの月きのふの月の中にけふ}\hfill{\FH 同}

    {\FK 昨日复明日长在月色中}
\end{haiku}

\begin{haiku}
    {\FH 尻を出し頭を出すや雲の月}\hfill{\FH 同}

    {\FK 明月云间隐露股又出头}
\end{haiku}

\begin{haiku}
    {\FH 海も山もくれてしまうて唯涼し}\hfill{\FH 同}

    {\FK 山海俱已暝沉沉唯觉凉}
\end{haiku}

\begin{haiku}
    {\FH 生きてゐるやうに動くや蓮の露}\hfill{\FH 同}

    {\FK 动摇莲叶上露滴生意滋}
\end{haiku}

\begin{haiku}
    {\FH よれよれに映る\ruby{手摺}{てすり}や春の水}\hfill{\FH 同}

    {\FK 曲曲倒影绉春水映画栏}
\end{haiku}

\begin{haiku}
    {\FH 夜も昼もうつらうつらと五月闇}\hfill{\FH 同}

    {\FK 五月晦暗甚昼夜似梦中}
\end{haiku}

\begin{haiku}
    {\FH 天は晴れ地は\ruby{湿}{うるお}ふや\ruby{鍬始}{くわはじめ}}\hfill{\FH 同}

    {\FK 天晴地亦润负锹始农耕}
\end{haiku}

\begin{haiku}
    {\FH 何として春の夕をまぎらさん}\hfill{\FH 同}

    {\FK 思量营何事排遣此春宵}
\end{haiku}

\begin{haiku}
    {\FH 蚊をたたくいそがはしさよ\ruby{写}{うつ}し物}\hfill{\FH 同}

    {\FK 伏案方书写炎暑扑蚊忙}
\end{haiku}

\begin{haiku}
    {\FH 姉が織り妹が縫うて更衣}\hfill{\FH 同}

    {\FK 姐织妹缝纫三春人更衣}
\end{haiku}

\begin{haiku}
    {\FH 春の夜やくらがり走る小提灯}\hfill{\FH 同}

    {\FK 提持行幽暗春夜小灯笼}
\end{haiku}

\begin{haiku}
    {\FH 柳桜柳桜と栽ゑにけり}\hfill{\FH 同}

    {\FK 柳樱复柳樱一一相间栽}
\end{haiku}

\begin{haiku}
    {\FH そのままに花を見た目を\ruby{瞑}{ふさ}がれぬ}\hfill{\FH 同}

    {\FK 悼静溪叟}

    {\FK 而今如此暝双目曾睹花}
\end{haiku}

\begin{haiku}
    {\FH 唐辛子かんで待つ夜の恨かな}\hfill{\FH 同}

    {\FK 咀嚼辣椒子候君夜恨长}
\end{haiku}

\begin{haiku}
    {\FH 蚤と蚊に一夜やせたる思ひかな}\hfill{\FH 同}

    {\FK 一夜人消瘦蚤蚊缠绕中}
\end{haiku}

\begin{haiku}
    {\FH 馬の\ruby{股}{また}ぬけつ\ruby{潜}{くぐ}りつ虻遊ぶ}\hfill{\FH 同}

    {\FK 潜入复出飏飞虻游马臀}
\end{haiku}

\begin{haiku}
    {\FH 夜の雨昼の嵐や置\ruby{炬燵}{こたつ}}\hfill{\FH 同}

    {\FK 严冬设暖炉夜雨还昼风}
\end{haiku}

\begin{haiku}
    {\FH 芋の用意酒の用意や人遅し}\hfill{\FH 同}

    {\FK 烹芋且备酒迟迟客不来}
\end{haiku}

\begin{haiku}
    {\FH 辛崎の松は枯れつつ茂りつつ}\hfill{\FH 同}

    {\FK 枯荣长往复辛崎植群松}
\end{haiku}

\begin{haiku}
    {\FH 寺見えて小道の曲る野菊かな}\hfill{\FH 同}

    {\FK 丛林遥可见曲径野菊开}
\end{haiku}

\begin{haiku}
    {\FH 紅葉折て夕日寒がる女かな}\hfill{\FH 同}

    {\FK 女郞折红叶夕阳送暮寒}
\end{haiku}

\begin{haiku}
    {\FH 秋淋し毛虫はひ行く石畳}\hfill{\FH 同}

    {\FK 石级秋寂寞毛虫爬行中}
\end{haiku}

\begin{haiku}
    {\FH 傘さして傾城なぶる春の雨}\hfill{\FH 同}

    {\FK 春雨倾城降拄伞行滂沱}
\end{haiku}

\begin{haiku}
    {\FH 冬の日の落ちて明るし城の松}\hfill{\FH 同}

    {\FK 冬日沉西去城松灿灿明}
\end{haiku}

\begin{haiku}
    {\FH 寒月や枯木の上の一つ星}\hfill{\FH 同}

    {\FK 迥临枯木上寒月一星高}
\end{haiku}

\begin{haiku}
    {\FH 夏帽や吹きとばされて\ruby{濠}{ほり}に落つ}\hfill{\FH 同}

    {\FK 夏帽风吹去飞落濠堑中}
\end{haiku}

\begin{haiku}
    {\FH 冬籠日記に夢をかきつける}\hfill{\FH 同}

    {\FK 日记述幻梦杜门守严冬}
\end{haiku}

\begin{haiku}
    {\FH 飯盗む狐追ふ声や麦の秋\footnote{\FT 此句和下一句是芜村所做,风格和子规有明显不同。猜测是译者在子规评论芜村的书中摘出。}}\hfill{\FH 蕪村}

    {\FK 尾追盗饭狐喧声动麦秋}
\end{haiku}

\begin{haiku}
    {\FH 戸を叩く狸と秋を惜しみけり}\hfill{\FH 同}

    {\FK 哀惜秋光尽野狸来叩门}
\end{haiku}

\begin{haiku}
    {\FH 見下せば灯の無き町の夜寒かな}\hfill{\FH 子規}

    {\FK 俯视无灯火村落生夜寒}
\end{haiku}

\begin{haiku}
    {\FH 薮入や思ひは同じ姉妹}\hfill{\FH 同}

    {\FK 姐妹省亲长感怀尽相同}
\end{haiku}

\begin{haiku}
    {\FH 雪の絵を春も掛けたる埃かな}\hfill{\FH 同}

    {\FK 绘卷写雪景春来亦承尘}
\end{haiku}

\begin{haiku}
    {\FH 北海や日蝕見えず昼の霧}\hfill{\FH 同}

    {\FK 日蚀不可见北海昼雾生}
\end{haiku}

\begin{haiku}
    {\FH 朝顔や気\ruby{儘}{まま}に咲いておもしろき}\hfill{\FH 同}

    {\FK 真有妙趣在牵牛随意开}
\end{haiku}

\begin{haiku}
    {\FH 凩に舞ひあがりたる落葉かな}\hfill{\FH 同}

    {\FK 飘飘高飞去落叶舞清风}
\end{haiku}

\begin{haiku}
    {\FH 春の風二つ帆のある小舟かな}\hfill{\FH 同}

    {\FK 荡漾春风里小舟扬双帆}
\end{haiku}

\begin{haiku}
    {\FH 春雨のわれ蓑着たり笠着たり}\hfill{\FH 同}

    {\FK 披蓑戴箬笠春雨落我身}
\end{haiku}

\begin{haiku}
    {\FH 砂の如き雲流れ行く朝の秋}\hfill{\FH 同}

    {\FK 浮云如砂细秋朝长天流}
\end{haiku}

\begin{haiku}
    {\FH 砂浜に足跡長き春日かな}\hfill{\FH 同}

    {\FK 阳春丽日照砂滨足印长}
\end{haiku}

\begin{haiku}
    {\FH 一桶の藍流しけり春の川}\hfill{\FH 同}

    {\FK 春日河川上一桶靛蓝流}
\end{haiku}

\begin{haiku}
    {\FH 秋風や侍町は\ruby{塀}{へい}ばかり}\hfill{\FH 同}

    {\FK 秋风过侍町唯见墙垣残}
\end{haiku}

\begin{haiku}
    {\FH 長き夜や人灯を取って庭を行く}\hfill{\FH 同}

    {\FK 人持灯火庭中去索莫三秋夜漫漫}
\end{haiku}

\begin{haiku}
    {\FH 短夜やうすものかかる銀屏風}\hfill{\FH 同}

    {\FK 银屏风里罗衣薄短夜于今成良宵}
\end{haiku}

\begin{haiku}
    {\FH 花に遠く桜に近し吉野川}\hfill{\FH 蕪村}

    {\FK 芳菲远放樱花近吉野川头美景收}
\end{haiku}

\begin{haiku}
    {\FH 妻や子や野営夢さめて雁の声}\hfill{\FH 子規}

    {\FK 偕妻携子野营去好梦却缘雁声回}
\end{haiku}

\begin{haiku}
    {\FH 長き夜や思ひ出すとき風が吹く}\hfill{\FH 同}

    {\FK 悠悠长宵寒风过思君容辉心欲摧}
\end{haiku}

\begin{haiku}
    {\FH 梅柳川に\ruby{臨}{のぞ}みて誰が楼ぞ}\hfill{\FH 同}

    {\FK 谁家楼台旁水筑梅柳成荫临岸栽}
\end{haiku}

\begin{haiku}
    {\FH 雪の跡木履草鞋の別れかな}\hfill{\FH 同}

    {\FK 雪中相送人已别木屐草鞋履痕留}
\end{haiku}

\begin{haiku}
    {\FH 巻葉上に高く\ruby{浮葉}{ふよう}下に広がる蓮や此時}\hfill{\FH 同}

    {\FK 卷叶舒展浮叶广轻莲此时怡客情}
\end{haiku}

\begin{haiku}
    {\FH 詩僧あり酒僧あり梅の園城寺}\hfill{\FH 同}

    {\FK 梅花开遍园城寺诗僧酒僧共骈田}
\end{haiku}

\begin{haiku}
    {\FH 歌書俳書紛然として昼寝かな}\hfill{\FH 同}

    {\FK 歌书句集纷然在日长人困梦邯郸}
\end{haiku}

\begin{haiku}
    {\FH 病起杖に倚れば\ruby{千山}{せんざん}\ruby{万岳}{ばんがく}の秋}\hfill{\FH 同}

    {\FK 病体差愈策杖望万岳千山尽秋光}
\end{haiku}

\begin{haiku}
    {\FH 生きて帰れ露の命と言ひながら}\hfill{\FH 同}

    {\FK 虽云薄命同朝露尚祈珍摄得生还}
\end{haiku}

\begin{haiku}
    {\FH 松陰はどこも銭出すあつさかな}\hfill{\FH 同}

    {\FK 赤日行天炎可畏随处松阴须解囊}
\end{haiku}

\begin{haiku}
    {\FH 筆も墨も\ruby{溲瓶}{しびん}も内に秋の蚊帳}\hfill{\FH 同}

    {\FK 秋来缠绵蚊帐里笔墨溲瓶长随身}
\end{haiku}

\begin{haiku}
    {\FH 蓬莱や南山の蜜柑東海の鰕}\hfill{\FH 同}

    {\FK 蓬莱台上珍物奢南山蜜柑东海虾}
\end{haiku}

\begin{haiku}
    {\FH 長病の今年も参る雑煮かな}\hfill{\FH 同}

    {\FK 长病今年喜小痊合家共进杂煮时}
\end{haiku}

\begin{haiku}
    {\FH 野良猫の糞して居るや冬の庭}\hfill{\FH 同}

    {\FK 冬日院落少人到野猫遗矢著当庭}
\end{haiku}

\begin{haiku}
    {\FH 五女ありて後の男や\ruby{初幟}{はつのぼり}}\hfill{\FH 同}

    {\FK 且喜添丁五女后屋上初展鲤鱼旗}
\end{haiku}

\begin{haiku}
    {\FH 酒売の夏川越えて岡越えて}\hfill{\FH 同}

    {\FK 酒保踯躅鬻杜康既越夏川亦度冈}
\end{haiku}

\begin{haiku}
    {\FH 花もなし実もなし枇杷の九月かな}\hfill{\FH 同}

    {\FK 九月枇杷叹凋敝芳实已尽花亦无}
\end{haiku}

\begin{haiku}
    {\FH 小夜時雨上野を虚子の來つつあらん}\hfill{\FH 同}

    {\FK 夜半候客时雨降虚子料自上野来}
\end{haiku}

\begin{haiku}
    {\FH 案内者も我等も濡れて花の雨}\hfill{\FH 同}

    {\FK 无端春时花雨落沾湿旅客导游人}
\end{haiku}

\begin{haiku}
    {\FH 茶屋もなく酒屋も見えず花一木}\hfill{\FH 同}

    {\FK 茶室酒馆无踪迹烂漫唯有一树花}
\end{haiku}

\begin{haiku}
    {\FH 桃梅を笑へば梅も桃を笑らふ}\hfill{\FH 同}

    {\FK 国会开幕}

    {\FK 五十偏知百步遥桃嘲梅兮梅笑桃}
\end{haiku}

\begin{haiku}
    {\FH 春や昔\ruby{古白}{こはく}といへる男あり}\hfill{\FH 同}

    {\FK 悼古白}

    {\FK 往昔有男名古白今春寂寥冥途赊}
\end{haiku}

\begin{haiku}
    {\FH 鷹狩や豫陽の太守武を好む}\hfill{\FH 同}

    {\FK 豫阳太守好武备严冬臂鹰围猎中}
\end{haiku}

\begin{haiku}
    {\FH 蜘の巣に蜘は留守也秋の風}\hfill{\FH 同}

    {\FK 扣关无人秋风里唯有蜘蛛守空巢}
\end{haiku}

\begin{haiku}
    {\FH 草花や名も無き小川水清し}\hfill{\FH 同}

    {\FK 一带草花盛开处无名小川流水清}
\end{haiku}

\begin{haiku}
    {\FH 芭蕉泣き蘇鉄は怒る野分かな}\hfill{\FH 同}

    {\FK 野风起时物态多芭蕉泪下苏铁怒}
\end{haiku}

\begin{haiku}
    {\FH 何も彼も水仙の水も新しき}\hfill{\FH 同}

    {\FK 贺新居}

    {\FK 万物无复旧模样水仙盆中水亦新}
\end{haiku}

\begin{haiku}
    {\FH 日一分一分ちゞまる冬至かな}\hfill{\FH 同}

    {\FK 白日一分一分减流年将逝冬至临}
\end{haiku}

\begin{haiku}
    {\FH \ruby{稲舟}{いなぶね}や野菊の渚\ruby{蓼}{たで}の岸}\hfill{\FH 同}

    {\FK 船载禾稻行何处野菊渚头蓼岸边}
\end{haiku}

\begin{haiku}
    {\FH 村は小春山は時雨と野の広さ}\hfill{\FH 同}

    {\FK 村墟小春山时雨一望平野广无俦}
\end{haiku}

\begin{haiku}
    {\FH どこまでも枯木と見せて梅の花}\hfill{\FH 同}

    {\FK 处处皆作枯木认谁识红梅艳开花}
\end{haiku}

\begin{haiku}
    {\FH \ruby{兜}{かぶと}脱げ酒ふるまはん鬢の霜}\hfill{\FH 同}

    {\FK 除却甲冑且酣饮烈士归来两鬓霜}
\end{haiku}

\begin{haiku}
    {\FH 椰子の陰に語れ牡丹を\ruby{芍薬}{しゃくやく}を}\hfill{\FH 同}

    {\FK 送南洋人归国}

    {\FK 椰子阴中作何语牡丹芍药入话题}
\end{haiku}

\begin{haiku}
    {\FH 気安さや花から花へ旅\ruby{乞食}{こじき}}\hfill{\FH 同}

    {\FK 沿途乞食意自若只在樱花胜处游}
\end{haiku}

\begin{haiku}
    {\FH 夜涼如水天の川辺の星一つ}\hfill{\FH 同}

    {\FK 夜凉如水秋宵里一星灿然银河边}
\end{haiku}

\begin{haiku}
    {\FH 暖爐焚くや玻璃窓外の風の松}\hfill{\FH 同}

    {\FK 室内御寒炉火盛玻璃窗外闻松涛}
\end{haiku}

\begin{haiku}
    {\FH 幾時雨石山の石に苔もなし}\hfill{\FH 同}

    {\FK 石山岩上苔不长曾经时雨落几回}
\end{haiku}

\begin{haiku}
    {\FH 木犀や母が教ふる二絃琴}\hfill{\FH 同}

    {\FK 木犀花发凉秋里从母学弹二弦琴}
\end{haiku}

\begin{haiku}
    {\FH 秋風に生れてさすが男かな}\hfill{\FH 同}

    {\FK 贺人得子}

    {\FK 终得男嗣传根脉喜看添丁秋风中}
\end{haiku}

\begin{haiku}
    {\FH 庭踏んで木の芽草の芽何度見る}\hfill{\FH 同}

    {\FK 步出庭中反复看草木迎春俱抽芽}
\end{haiku}

\begin{haiku}
    {\FH 鳩鳴くや大提灯の春の風}\hfill{\FH 同}

    {\FK 浅草}

    {\FK 鸠鸣东陆佳气至大灯笼上过春风}
\end{haiku}

\begin{haiku}
    {\FH 冬枯や巡査に吠ゆる里の犬}\hfill{\FH 同}

    {\FK 严冬肃杀警察至引得犬吠闾巷中}
\end{haiku}

\begin{haiku}
    {\FH 通夜堂や緑の中の\ruby{百日紅}{さるすべり}}\hfill{\FH 同}

    {\FK 通夜堂前风物佳万绿丛中百日红}
\end{haiku}

\begin{haiku}
    {\FH 穴にのぞく余寒の蟹の\ruby{爪}{つま}赤し}\hfill{\FH 同}

    {\FK 螃蟹现赤爪窥穴觉余寒}
\end{haiku}

\begin{haiku}
    {\FH \ruby{痩骨}{そうこつ}をさする朝寒夜寒かな}\hfill{\FH 同}

    {\FK 朝寒夜寒重瘦骨抚摸中}
\end{haiku}

\begin{haiku}
    {\FH 病床のうめきに和して秋の蝉}\hfill{\FH 同}

    {\FK 病床发呻吟秋蝉酬答中}
\end{haiku}

\begin{haiku}
    {\FH 草鞋はいて木曾路の露につまづくな}\hfill{\FH 同}

    {\FK 足登草鞋勿颠簸木曾路上清露多}
\end{haiku}

\begin{haiku}
    {\FH 糸瓜咲て痰のつまりし仏かな}\hfill{\FH 同}

    {\FK 绝笔}

    {\FK 浓痰壅塞命如丝正值丝瓜初开时}
\end{haiku}

\begin{haiku}
    {\FH 痰一斗糸瓜の水も間にあはず}\hfill{\FH 同}

    {\FK 其二}

    {\FK 清凉纵如丝瓜汁难疗喉头一斗痰}
\end{haiku}

\begin{haiku}
    {\FH をととひの糸瓜の水も取らざりき}\hfill{\FH 同}

    {\FK 其三}

    {\FK 前日丝瓜正鲜嫩忘取清液疗病身}
\end{haiku}

\begin{haiku}
    {\FH 燭剪って見守る太刀や夜半の秋}\hfill{\FH 松宇}

    {\FK 大刀剪烛视凉秋午夜中}
\end{haiku}

\begin{haiku}
    {\FH 雄大な句を思う夜の野分かな}\hfill{\FH 同}

    {\FK 思咏磅礴句入夜凄厉风}
\end{haiku}

\begin{haiku}
    {\FH 埋火や昼小暗きに恋歌よむ}\hfill{\FH 洒竹}

    {\FK 白昼微黝暗火边读恋歌}
\end{haiku}

\begin{haiku}
    {\FH 峰越せば雨にちりけり春の雪}\hfill{\FH 醒雪}

    {\FK 春雪越峰过消散雨丝中}
\end{haiku}

\begin{haiku}
    {\FH もの思う傾城老いぬ夕桜}\hfill{\FH 同}

    {\FK 美色倾城今已老回首万事夕樱开}
\end{haiku}

\begin{haiku}
    {\FH 半日の車の旅や蝉に飽く}\hfill{\FH 瓊音}

    {\FK 旅途半日车中过双耳饱闻鸣蝉声}
\end{haiku}

\begin{haiku}
    {\FH 蝶ひらひら天下の春をほしいまま}\hfill{\FH 竹冷}

    {\FK 天下春自在蝴蝶疾飞中}
\end{haiku}

\begin{haiku}
    {\FH 竹林や雨来りそそぎ蛍活く}\hfill{\FH 同}

    {\FK 却暑夏雨注竹林萤竞飞}
\end{haiku}

\begin{haiku}
    {\FH 草いろいろ我恋ふ野菊しほらしや}\hfill{\FH 同}

    {\FK 芳草万千色各殊心恋野菊娇媚姿}
\end{haiku}

\begin{haiku}
    {\FH 新曲や春の句二つ三つよせて}\hfill{\FH 知十}

    {\FK 新曲动诗兴春句咏二三}
\end{haiku}

\begin{haiku}
    {\FH 桜咲く日本に生まれ男かな}\hfill{\FH 小波}

    {\FK 男儿生日本樱花春烂漫}
\end{haiku}

\begin{haiku}
    {\FH \ruby{沃野}{よくや}千里露万斛のあしたかな}\hfill{\FH 同}

    {\FK 明日万斛露沃野千里中}
\end{haiku}

\begin{haiku}
    {\FH 若干の詩債もありて冬籠}\hfill{\FH 同}

    {\FK 严冬闭门坐诗债积累多}
\end{haiku}

\begin{haiku}
    {\FH 虻一つ遊子\ruby{悶}{もだえ}の窓をうつ}\hfill{\FH 同}

    {\FK 游子郁闷坐室内一头飞蝇来扑窗}
\end{haiku}

\begin{haiku}
    {\FH 桃満村白きもの只水なりけり}\hfill{\FH 黄雨}

    {\FK 唯有水色白满村桃花红}
\end{haiku}

\begin{haiku}
    {\FH 霜の鐘古陵の松に答えけり}\hfill{\FH 同}

    {\FK 洪钟凌霜动应酬古陵松}
\end{haiku}

\begin{haiku}
    {\FH \ruby{草廬}{そうろ}五尺にして元日の心大いなり}\hfill{\FH 同}

    {\FK 五尺草庐内元日意气豪}
\end{haiku}

\begin{haiku}
    {\FH \ruby{禿}{ち}びけりな\ruby{筆}{ふで}も箒も年の果}\hfill{\FH 同}

    {\FK 笔箒倶秃败行看岁运阑}
\end{haiku}

\begin{haiku}
    {\FH 春の水十里家なき郷を過ぐ}\hfill{\FH 同}

    {\FK 十里春水过乡间无人家}
\end{haiku}

\begin{haiku}
    {\FH 縁蔭に昔をしのぶ\ruby{礎石}{そせき}あり}\hfill{\FH 紅葉}

    {\FK 教忆旧日事石础绿荫中}
\end{haiku}

\begin{haiku}
    {\FH 春光や浪の揺ぎにゆらぐ鳥}\hfill{\FH 麦人}

    {\FK 春光波上照鸟随浪动摇}
\end{haiku}

\begin{haiku}
    {\FH 吸いつくや袖に袂に春の雪}\hfill{\FH 同}

    {\FK 吸附袖袂上春雪落霏微}
\end{haiku}

\begin{haiku}
    {\FH 蟻の行く音蜘の走る音す古草叢}\hfill{\FH 同}

    {\FK 蚁行蜘蛛走声发古草丛}
\end{haiku}

\begin{haiku}
    {\FH 夜の底に沈む街ただ稲妻す}\hfill{\FH 鶯塘}

    {\FK 唯睹电光影街寂夜深沉}
\end{haiku}

\begin{haiku}
    {\FH 金魚斃ちて鉢にさびしき浮藻かな}\hfill{\FH 同}

    {\FK 钵中金鱼已死绝浮萍残留觉寂寥}
\end{haiku}

\begin{haiku}
    {\FH 雨音の中に老いたり虫の声}\hfill{\FH 畊石}

    {\FK 入秋鸣寒虫人老雨声中}
\end{haiku}

\begin{haiku}
    {\FH 暁の雲菊の香に\ruby{仰}{あお}ぐなり}\hfill{\FH 同}

    {\FK 令我频仰首晓云与菊香}
\end{haiku}

\begin{haiku}
    {\FH 屠蘇に酔う男と我を人知らじ}\hfill{\FH 同}

    {\FK 旁人无有识我者唯知此男醉屠苏}
\end{haiku}

\begin{haiku}
    {\FH 何の木か寒の夜風にそそり立つ}\hfill{\FH 同}

    {\FK 未识是何树挺立寒夜风}
\end{haiku}

\begin{haiku}
    {\FH 茄子のむらさき唐辛子の朱や渾て露}\hfill{\FH 五丈原}

    {\FK 尽教凉露浸润遍茄子色紫辣椒红}
\end{haiku}

\begin{haiku}
    {\FH 春動く見よかぜぐるま水ぐるま}\hfill{\FH 和風}

    {\FK 风车水车转遂识春动摇}
\end{haiku}

\begin{haiku}
    {\FH 狂人か預言者か枯野風にさけぶ}\hfill{\FH 同}

    {\FK 狂人抑或预言者是谁枯野号寒风}
\end{haiku}

\begin{haiku}
    {\FH 一諾を得て帰えり行く夜長人}\hfill{\FH 素琴}

    {\FK 邀得一诺后人归长夜间}
\end{haiku}

\begin{haiku}
    {\FH 三本の柿に秋光湛えたり}\hfill{\FH 同}

    {\FK 三株柿树上秋光何湛湛}
\end{haiku}

\begin{haiku}
    {\FH 石に踞して秋の雲見る無関心}\hfill{\FH 晋風}

    {\FK 踞石望秋云万事不关心}
\end{haiku}

\begin{haiku}
    {\FH 電車近く枯原に影走らせる}\hfill{\FH 同}

    {\FK 电车行渐近投影走荒原}
\end{haiku}

\begin{haiku}
    {\FH 医の下手詩のなお下手の菊つくり}\hfill{\FH 紫影}

    {\FK 诗作更较医道劣秋来殷勤培黄花}
\end{haiku}

\begin{haiku}
    {\FH 平沙千里駱駝の背から雲の峰}\hfill{\FH 水哉}

    {\FK 骆驼背上云峰下平沙千里入望中}
\end{haiku}

\begin{haiku}
    {\FH 涼しさや一万二千峰の風}\hfill{\FH 臨風}

    {\FK 朝鲜金刚山}

    {\FK 凉风猎猎起苹末吹遍一万二千峰}
\end{haiku}

\begin{haiku}
    {\FH 元朝やまず大杯に酒盛らむ}\hfill{\FH 迂外}

    {\FK 大杯先倾酒元旦意气豪}
\end{haiku}

\begin{haiku}
    {\FH 元日や一糸の天子不二の山}\hfill{\FH 鳴雪}

    {\FK 欣逢元朝至心中多感慨皇统总一系不二名岳在}
\end{haiku}

\begin{haiku}
    {\FH 千万里其行先も春の風}\hfill{\FH 同}

    {\FK 征途千万里前程亦春风}
\end{haiku}

\begin{haiku}
    {\FH 古道に梅の一枝の余寒かな}\hfill{\FH 同}

    {\FK 古道一枝梅梢头着余寒}
\end{haiku}

\begin{haiku}
    {\FH 夏羽織白き単衣を愛すかな}\hfill{\FH 同}

    {\FK 夏日罗衫薄心爱白单衣}
\end{haiku}

\begin{haiku}
    {\FH 朝立や馬のかしらの天の川}\hfill{\FH 同}

    {\FK 银河马头上晓发天未明}
\end{haiku}

\begin{haiku}
    {\FH 横たはる五尺の\ruby{榾}{ほた}やちょろちょろ火}\hfill{\FH 同}

    {\FK 毕剥炉火盛五尺榾柚横}
\end{haiku}

\begin{haiku}
    {\FH 凩の吹き荒るる中の午砲かな}\hfill{\FH 同}

    {\FK 寒风此时烈午炮传响时}
\end{haiku}

\begin{haiku}
    {\FH 只頼む湯婆一つの寒さかな}\hfill{\FH 同}

    {\FK 绝笔}

    {\FK 残躯唯赖汤婆煖病身不堪风霜欺}
\end{haiku}

\begin{haiku}
    {\FH 青蛙孤峰の雨をよばうかな}\hfill{\FH 鬼城}

    {\FK 青蛙啼何苦似呼孤峰雨}
\end{haiku}

\begin{haiku}
    {\FH 禿山やはためきかえす日雷}\hfill{\FH 同}

    {\FK 殷殷响日雷回声绕秃山}
\end{haiku}

\begin{haiku}
    {\FH うとうとと生死の外や日向ぼこ}\hfill{\FH 同}

    {\FK 木然生死外冬日负曝情}
\end{haiku}

\begin{haiku}
    {\FH 行春や親になりたる盲犬}\hfill{\FH 同}

    {\FK 匆匆春归去盲犬成亲人}
\end{haiku}

\begin{haiku}
    {\FH 沼涸れて狼渡る月夜かな}\hfill{\FH 同}

    {\FK 凉秋月色好贪狼渡涸沼}
\end{haiku}

\begin{haiku}
    {\FH 名香に酔いて春の夜ねむられず}\hfill{\FH 同}

    {\FK 名香令人醉春夜无眠中}
\end{haiku}

\begin{haiku}
    {\FH 春寒やぶつかり歩く盲犬}\hfill{\FH 同}

    {\FK 碰壁犹蹒跚盲犬步春寒}
\end{haiku}

\begin{haiku}
    {\FH 雹晴れて\ruby{豁然}{かつぜん}とある山河かな}\hfill{\FH 同}

    {\FK 雹落放晴后豁然现山河}
\end{haiku}

\begin{haiku}
    {\FH 春の夜や灯を\ruby{囲}{かこ}み居る盲者達}\hfill{\FH 同}

    {\FK 围聚灯火坐群盲春夜中}
\end{haiku}

\begin{haiku}
    {\FH 世を恋うて人を恐るる余寒かな}\hfill{\FH 同}

    {\FK 惧人恋尘世冬尽怯余寒}
\end{haiku}

\begin{haiku}
    {\FH 秋山に僧と\ruby{携}{たずさ}ふ詩盟かな}\hfill{\FH 同}

    {\FK 同志结诗盟携僧游秋山}
\end{haiku}

\begin{haiku}
    {\FH 苔咲くや親にわかれて二十年}\hfill{\FH 同}

    {\FK 忍见莓苔花又放别亲于兹二十年}
\end{haiku}

\begin{haiku}
    {\FH 霍乱や里に一人の盲医者}\hfill{\FH 同}

    {\FK 里中霍乱方肆虐盲医一人来回春}
\end{haiku}

\begin{haiku}
    {\FH \ruby{治聾酒}{じろうしゅ}の酔ふほどもなくさめにけり}\hfill{\FH 同}

    {\FK 纵令酣饮亦清醒酒能治聋不醉人}
\end{haiku}

\begin{haiku}
    {\FH \ruby{乞食}{こつじき}を\ruby{葬}{ほうぶ}る月の光かな}\hfill{\FH 古白}

    {\FK 乞儿今命绝葬汝月光来}
\end{haiku}

\begin{haiku}
    {\FH 草の戸の我に溢るる初日かな}\hfill{\FH 飄亭}

    {\FK 茅舍初日照朝阳溢我身}
\end{haiku}

\begin{haiku}
    {\FH \ruby{酒気}{しゅき}\ruby{白虹}{はっこう}の如く寒月を\ruby{貫}{つらぬ}けり}\hfill{\FH 同}

    {\FK 冲天贯寒月酒气如白虹}
\end{haiku}

\begin{haiku}
    {\FH 天下\ruby{猶}{なお}取り得ず独り蝿を打つ}\hfill{\FH 同}

    {\FK 独自拍蝇暑日里天下犹未归掌中}
\end{haiku}

\begin{haiku}
    {\FH 谷底へ朽し落ちし堂や閑古鳥}\hfill{\FH 月斗}

    {\FK 废堂沉谷底空山郭公啼}
\end{haiku}

\begin{haiku}
    {\FH 久方の月の\ruby{仄}{ほの}かに春の雨}\hfill{\FH 同}

    {\FK 总为春雨落良月久朦胧}
\end{haiku}

\begin{haiku}
    {\FH \ruby{飄}{ひょう}々と風に一羽や寒鴉}\hfill{\FH 同}

    {\FK 飘飘随风去一头寒鸦飞}
\end{haiku}

\begin{haiku}
    {\FH 山の灯の消えてはとぼる野分かな}\hfill{\FH 同}

    {\FK 凉秋疾风动山灯消又明}
\end{haiku}

\begin{haiku}
    {\FH 虫の中に寝てしまひたる小村かな}\hfill{\FH 同}

    {\FK 人家尽入梦小村虫声中}
\end{haiku}

\begin{haiku}
    {\FH 黒\ruby{南風}{はえ}に\ruby{鱗}{うろくづ}匂ふ漁村かな}\hfill{\FH 同}

    {\FK 梅雨来时南风暗吹送渔村鳞介香}
\end{haiku}

\begin{haiku}
    {\FH 亡き妻の鏡に春の寒さかな}\hfill{\FH 肋骨}

    {\FK 悼亡}

    {\FK 发妻赴泉路明镜生春寒}
\end{haiku}

\begin{haiku}
    {\FH 昼寝ざめ大事去りたる西日かな}\hfill{\FH 青峰}

    {\FK 大事去矣不可为昼寝觉来日西沉}
\end{haiku}

\begin{haiku}
    {\FH 幾秋の泉を旅の鏡かな}\hfill{\FH 露月}

    {\FK 旅途泉为镜如此经几秋}
\end{haiku}

\begin{haiku}
    {\FH 墨の痕と泉の声とけさの秋}\hfill{\FH 同}

    {\FK 凉秋今朝至墨迹泉声留}
\end{haiku}

\begin{haiku}
    {\FH 草枯や一\ruby{夢}{む}と消えし都の灯}\hfill{\FH 同}

    {\FK 严冬沍寒草枯死都会灯如一梦消}
\end{haiku}

\begin{haiku}
    {\FH 話尽きて湖の夕立\ruby{瞰}{み}\ruby{下}{おろ}しけり}\hfill{\FH 霽月}

    {\FK 入吾室者但有清风}

    {\FK 语尽今俯瞰湖上白雨骤}
\end{haiku}

\begin{haiku}
    {\FH 寝てきくや\ruby{深}{み}山木にふる春の雨}\hfill{\FH 同}

    {\FK 浊酒聊自适鼓腹无所思}

    {\FK 洒落深山树卧听春雨声}
\end{haiku}

\begin{haiku}
    {\FH 窓明りすれば君ある夜寒かな}\hfill{\FH 梧月}

    {\FK 窗明知君在隆冬识夜寒}
\end{haiku}

\begin{haiku}
    {\FH 秋風や故人\ruby{用}{もち}いし穂長\ruby{筆}{ふで}}\hfill{\FH 同}

    {\FK 子规庵}

    {\FK 遗物膽仰秋风里故人曾用长锋毫}
\end{haiku}

\begin{haiku}
    {\FH 花びらを曲げて明日なき野菊かな}\hfill{\FH 一転}

    {\FK 野菊明日尽花萼屈曲中}
\end{haiku}

\begin{haiku}
    {\FH 秋風や\ruby{莟}{つぼみ}も持たず\ruby{薔薇}{そうび}の芽}\hfill{\FH 同}

    {\FK 九十秋风萧然过蔷薇抽芽未含苞}
\end{haiku}

\begin{haiku}
    {\FH 昨日の事がゆめのような春暁の雨}\hfill{\FH 露石}

    {\FK 绝笔}

    {\FK 咋日事如梦春晓雨霏微}
\end{haiku}

\begin{haiku}
    {\FH 恋猫の月に尾あげし長さかな}\hfill{\FH 蕪子}

    {\FK 春猫恋情动举尾映月长}
\end{haiku}

\begin{haiku}
    {\FH 夏帽や相\ruby{顧}{かえりみ}る日本人}\hfill{\FH 同}

    {\FK 于西雅图}

    {\FK 夏帽相回顾倶是日本人}
\end{haiku}

\begin{haiku}
    {\FH 春の夜蛇の\ruby{紋}{あや}ある圓\ruby{柱}{はしら}}\hfill{\FH 同}

    {\FK 巴黎歌剧院}

    {\FK 圆柱蛇纹现春夜盘曲中}
\end{haiku}

\begin{haiku}
    {\FH \ruby{眺}{なが}めいる日南に雨気や春の山}\hfill{\FH 青青}

    {\FK 纵目望日南雨气遍春山}
\end{haiku}

\begin{haiku}
    {\FH 霜に飽く\ruby{黄葉}{こうよう}村の垣根かな}\hfill{\FH 同}

    {\FK 村家垣根落黄叶饱严霜}
\end{haiku}

\begin{haiku}
    {\FH 春愁やこの身このまま旅ごころ}\hfill{\FH 頼江}

    {\FK 我身长如此春愁牵旅情}
\end{haiku}

\begin{haiku}
    {\FH 思いきや新酒に君が不平ある}\hfill{\FH 插雲}

    {\FK 岂料新酒进使我怀不平}
\end{haiku}

\begin{haiku}
    {\FH 腸に春滴るや粥の味}\hfill{\FH 漱石}

    {\FK 粥味滴滴佳腸中春欲苏}
\end{haiku}

\begin{haiku}
    {\FH 霧黄なる市に動くや影法師}\hfill{\FH 同}

    {\FK 于伦敦闻子规讣报}

    {\FK 都市笼黄雾朦胧影动摇}
\end{haiku}

\begin{haiku}
    {\FH 初冬や竹伐る山の鉈の音}\hfill{\FH 同}

    {\FK 初冬伐竹去斧斤响山头}
\end{haiku}

\begin{haiku}
    {\FH 世に人あり枯野に石のありにけり}\hfill{\FH 東洋城}

    {\FK 世间凡夫在枯野顽石多}
\end{haiku}

\begin{haiku}
    {\FH 市に出ずや行人征馬春借む}\hfill{\FH 同}

    {\FK 行人征马出市去悠悠无限惜春情}
\end{haiku}

\begin{haiku}
    {\FH 口笛や春谷へだて知らぬ同士}\hfill{\FH 虚吼}

    {\FK 口哨声传识同好人隔春谷不相知}
\end{haiku}

\begin{haiku}
    {\FH 庵に在りて風瓢々の夏衣}\hfill{\FH 碧梧桐}

    {\FK 夏日庵中在单衣舞轻风}
\end{haiku}

\begin{haiku}
    {\FH 赤い椿白い椿と落ちにけり}\hfill{\FH 同}

    {\FK 色泽红或白山茶落纷纷}
\end{haiku}

\begin{haiku}
    {\FH 鶴とんでかえらず池の水寒}\hfill{\FH 同}

    {\FK 翔鹤去不返空余池水寒}
\end{haiku}

\begin{haiku}
    {\FH 蟹とれば蝦も手に飛ぶ涼しさよ}\hfill{\FH 同}

    {\FK 捉蟹惊虾起拂手凉意生}
\end{haiku}

\begin{haiku}
    {\FH 曲すみし笛の余音や春の月}\hfill{\FH 同}

    {\FK 余音春月下一曲横笛终}
\end{haiku}

\begin{haiku}
    {\FH ちり\ruby{布}{し}きし桃の上に雨の音あらん}\hfill{\FH 同}

    {\FK 题某妓三弦}

    {\FK 散布夭桃上似有雨滴声}
\end{haiku}

\begin{haiku}
    {\FH 野の家の桃に垣してとなり同志}\hfill{\FH 同}

    {\FK 乡野桃为垣比邻同好家}
\end{haiku}

\begin{haiku}
    {\FH 水月てる\ruby{疾}{と}き魚もみゆる風かおる}\hfill{\FH 同}

    {\FK 风熏月照水得见疾游鱼}
\end{haiku}

\begin{haiku}
    {\FH 愕然として昼寝さめたる一人かな}\hfill{\FH 同}

    {\FK 愕然昼寐觉孑立一身零}
\end{haiku}

\begin{haiku}
    {\FH 一宿して雨はるる山ひやひやと}\hfill{\FH 同}

    {\FK 一宿雨收后山晴丘壑凉}
\end{haiku}

\begin{haiku}
    {\FH 火も置かず独居の人と夜長かな}\hfill{\FH 同}

    {\FK 夜长不置火索然人独居}
\end{haiku}

\begin{haiku}
    {\FH 月の雨静かに雨を聞く夜かな}\hfill{\FH 同}

    {\FK 月夜悄然雨静听淅沥声}
\end{haiku}

\begin{haiku}
    {\FH 一器の食明日を思うや宵の冬}\hfill{\FH 同}

    {\FK 冬宵食一器明日入我思}
\end{haiku}

\begin{haiku}
    {\FH 送別の月寒く酒を強いにけり}\hfill{\FH 同}

    {\FK 强饮离前酒送别月寒中}
\end{haiku}

\begin{haiku}
    {\FH 何処来て里に落ちたる吹雪かな}\hfill{\FH 同}

    {\FK 不知来何处吹雪落此乡}
\end{haiku}

\begin{haiku}
    {\FH 帰り来ぬ人北風に立つ日かな}\hfill{\FH 同}

    {\FK 人自远方至此日立北风}
\end{haiku}

\begin{haiku}
    {\FH 句はなくて妓を品す菊の花の前}\hfill{\FH 同}

    {\FK 欲咏未得句品妓菊花前}
\end{haiku}

\begin{haiku}
    {\FH 極月の風雪熊羆叫びけり}\hfill{\FH 同}

    {\FK 岁暮风雪紧熊罴叫严寒}
\end{haiku}

\begin{haiku}
    {\FH 再遊の宿も同じき砧かな}\hfill{\FH 同}

    {\FK 再游宿此处砧声旧时同}
\end{haiku}

\begin{haiku}
    {\FH 桃咲くや湖水のへりの十個村}\hfill{\FH 同}

    {\FK 桃花春开艳十村湖水边}
\end{haiku}

\begin{haiku}
    {\FH さかのぼりて君を迎えぬ春の水}\hfill{\FH 同}

    {\FK 沿溯上春水迎君不辞劳}
\end{haiku}

\begin{haiku}
    {\FH 海楼の涼しさつひの別れかな}\hfill{\FH 同}

    {\FK 而今终成别海楼凜凛凉}
\end{haiku}

\begin{haiku}
    {\FH 行く蛍白雲洞の道を照らす}\hfill{\FH 同}

    {\FK 径通白云洞流萤助照明}
\end{haiku}

\begin{haiku}
    {\FH 谷水の地底になりて夜さむかな}\hfill{\FH 同}

    {\FK 冬夜寒气重谷水地底流}
\end{haiku}

\begin{haiku}
    {\FH 夜を寒み人語聞えて森の寺}\hfill{\FH 同}

    {\FK 夜寒闻人语兰若木森森}
\end{haiku}

\begin{haiku}
    {\FH 遠のきし雲夕栄えす秋の山}\hfill{\FH 同}

    {\FK 秋山向晚艳远方浮云来}
\end{haiku}

\begin{haiku}
    {\FH 軒落ちて雪窮巷を塞ぎけり}\hfill{\FH 同}

    {\FK 轩前积雪落穷巷壅不通}
\end{haiku}

\begin{haiku}
    {\FH 寒稽古子弟の骨を\ruby{鍛}{きた}へけり}\hfill{\FH 同}

    {\FK 子弟炼筋骨寒冬勤勿怠}
\end{haiku}

\begin{haiku}
    {\FH 雲こめし中や雨ふる秋の山}\hfill{\FH 同}

    {\FK 秋山云缭绕雨落洒其中}
\end{haiku}

\begin{haiku}
    {\FH 豊かなる年の落ち穂を祝いけり}\hfill{\FH 同}

    {\FK 年丰五谷实落穗祝拜中}
\end{haiku}

\begin{haiku}
    {\FH 凩や皆くねりたる磯の松}\hfill{\FH 同}

    {\FK 雪折寒风里群松立矶头}
\end{haiku}

\begin{haiku}
    {\FH 大風に傷みし木々や渡り鳥}\hfill{\FH 同}

    {\FK 候鸟归南去大风万木伤}
\end{haiku}

\begin{haiku}
    {\FH 萎んだミモ一サの花の色衰えず}\hfill{\FH 同}

    {\FK 含羞花已谢艳色犹未衰}
\end{haiku}

\begin{haiku}
    {\FH 酒を置いて老の涙の火桶かな}\hfill{\FH 同}

    {\FK 残年洒老泪置酒火钵边}
\end{haiku}

\begin{haiku}
    {\FH 砲車過ぐる巷の塵や日の盛り}\hfill{\FH 同}

    {\FK 炮车经过闾巷里尘沙飞舞日照强}
\end{haiku}

\begin{haiku}
    {\FH 雪を渡りて又薫風の草花踏む}\hfill{\FH 同}

    {\FK 才得行经积雪上又踏草花熏风中}
\end{haiku}

\begin{haiku}
    {\FH 近作を画室に掛くる秋日影}\hfill{\FH 同}

    {\FK 近作悬挂画室里三秋日光入户来}
\end{haiku}

\begin{haiku}
    {\FH 故人ここに在りし遺物と新酒かな}\hfill{\FH 同}

    {\FK 怀子规居士旧事}

    {\FK 故人往昔曾在此遺物新酒感触多}
\end{haiku}

\begin{haiku}
    {\FH 寒林貧寺焼けたり僧の留守}\hfill{\FH 同}

    {\FK 衲子不知何处去寒林贫寺逢祝融}
\end{haiku}

\begin{haiku}
    {\FH 墓と見えて十字架立つる秋の山}\hfill{\FH 同}

    {\FK 十字架立秋山上嶕峣高坟入望中}
\end{haiku}

\begin{haiku}
    {\FH 難行を天下に誇るさむさかな}\hfill{\FH 同}

    {\FK 缯纩无温寒气重敢夸天下行路难}
\end{haiku}

\begin{haiku}
    {\FH 大戦に死所を得ず哀れ野はかれて}\hfill{\FH 同}

    {\FK 读史}

    {\FK 塵战未获埋骨地哀哉寒野尽萧条}
\end{haiku}

\begin{haiku}
    {\FH 枯木原に雪積んでいる月夜かな}\hfill{\FH 同}

    {\FK 夜月当空照此景原野雪封万木凋}
\end{haiku}

\begin{haiku}
    {\FH 手をかざせば睡魔の襲ふ火桶かな}\hfill{\FH 同}

    {\FK 睡魔而今来袭我伸手向火取暖时}
\end{haiku}

\begin{haiku}
    {\FH 山囲む\ruby{帰臥}{きが}の天地や柿の秋}\hfill{\FH 同}

    {\FK 柿实成熟三秋里归卧天地万山中}
\end{haiku}

\begin{haiku}
    {\FH 天下知る蔵書見に来ぬ秋の晴れ}\hfill{\FH 同}

    {\FK 藏书名高传天下来求披览趁秋晴}
\end{haiku}

\begin{haiku}
    {\FH 後宮の夜半に雪折きこえけり}\hfill{\FH 同}

    {\FK 枝柯断折积雪重后宫夜半闻此声}
\end{haiku}

\begin{haiku}
    {\FH 大樹の下\ruby{児女}{じじょ}\ruby{鶏犬}{けいけん}に風薫る}\hfill{\FH 同}

    {\FK 儿女鸡犬大树下南风过处送熏香}
\end{haiku}

\begin{haiku}
    {\FH 谷深うまこと一人や\ruby{漆}{うるし}\ruby{掻}{か}き}\hfill{\FH 同}

    {\FK 深谷迢遥长漆树剥皮取液唯一人}
\end{haiku}

\begin{haiku}
    {\FH 死後の知己を生前に秋晴るるなり}\hfill{\FH 同}

    {\FK 死后知己生前遇飒爽三秋晴和天}
\end{haiku}

\begin{haiku}
    {\FH 春雷のひびきわたりし\ruby{館}{やかた}かな}\hfill{\FH 虚子}

    {\FK 平地春雷震栏槛传惊霆}
\end{haiku}

\begin{haiku}
    {\FH \ruby{芳草}{ほうそう}や黒き烏も\ruby{濃}{こ}紫}\hfill{\FH 同}

    {\FK 黑乌亦浓紫芳草何芊眠}
\end{haiku}

\begin{haiku}
    {\FH 各各の鬢霜のおく春さむし}\hfill{\FH 同}

    {\FK 鬓发各已斑相与畏春寒}
\end{haiku}

\begin{haiku}
    {\FH 晩涼に池の萍みな動く}\hfill{\FH 同}

    {\FK 风来萍皆动临池追晚凉}
\end{haiku}

\begin{haiku}
    {\FH 老の\ruby{頬}{ほ}に\ruby{紅潮}{くれないさ}すや濁酒}\hfill{\FH 同}

    {\FK 浊醪力不胜老颊生红潮}
\end{haiku}

\begin{haiku}
    {\FH \ruby{帰雁}{きがん}あり吾に辞任の心あり}\hfill{\FH 同}

    {\FK 仰见归雁过长愿辞宦游}
\end{haiku}

\begin{haiku}
    {\FH 紅葉濃し池に映りて更に濃し}\hfill{\FH 同}

    {\FK 红叶本如火映池色更浓}
\end{haiku}

\begin{haiku}
    {\FH 病葉や大地に何の病ある}\hfill{\FH 同}

    {\FK 大地染何疾病叶遍尔身}
\end{haiku}

\begin{haiku}
    {\FH 一つ根に離れ浮く葉や春の水}\hfill{\FH 同}

    {\FK 落叶辞根去长随春水流}
\end{haiku}

\begin{haiku}
    {\FH 菊は日に\ruby{驕}{おご}り芒は霜に伏し}\hfill{\FH 同}

    {\FK 芒随严霜伏菊逐阳煦骄}
\end{haiku}

\begin{haiku}
    {\FH 夕嵐青鷺吹き去って高楼に灯}\hfill{\FH 同}

    {\FK 漠漠天色昏夕风遒且劲青鹭随风去高楼灯火明}
\end{haiku}

\begin{haiku}
    {\FH 初鏡にはびこりうつる華鬢かな}\hfill{\FH 同}

    {\FK 新岁初临镜感叹霜鬂加}
\end{haiku}

\begin{haiku}
    {\FH 寒菊や年年同じ庭の隅}\hfill{\FH 同}

    {\FK 寒菊长如此年年立庭隅}
\end{haiku}

\begin{haiku}
    {\FH \ruby{旗竿}{はたざお}に流る春光学校旗}\hfill{\FH 同}

    {\FK 校旗髙悬处杆头春光流}
\end{haiku}

\begin{haiku}
    {\FH 水かえて金魚の金に水の銀}\hfill{\FH 同}

    {\FK 银色水波动灿烂金鱼游}
\end{haiku}

\begin{haiku}
    {\FH ふるさとの月の\ruby{港}{みなと}をよぎるのみ}\hfill{\FH 同}

    {\FK 夜船过旧港月是故乡明}
\end{haiku}

\begin{haiku}
    {\FH 牡丹花の面影のこし崩れけり}\hfill{\FH 同}

    {\FK 悼橙黄子}

    {\FK 玉容似牡丹人去面影残}
\end{haiku}

\begin{haiku}
    {\FH 花と葉と\ruby{相搏}{あいう}つ風の椿かな}\hfill{\FH 同}

    {\FK 山茶乘春荣花叶互搏风}
\end{haiku}

\begin{haiku}
    {\FH 手を上げて別るる時の春の月}\hfill{\FH 同}

    {\FK 挥手自兹去劳劳春月中}
\end{haiku}

\begin{haiku}
    {\FH 仲秋や\ruby{互}{かたみ}にひろき海と陸}\hfill{\FH 同}

    {\FK 仲秋舒望眼海广陆亦宽}
\end{haiku}

\begin{haiku}
    {\FH 来ん秋を再び期して別れけり}\hfill{\FH 同}

    {\FK 今日此为别来秋期重逢}
\end{haiku}

\begin{haiku}
    {\FH 唄いつつ\ruby{笑}{え}まいつつ行く春の人}\hfill{\FH 同}

    {\FK 春来人皆喜载歌载笑行}
\end{haiku}

\begin{haiku}
    {\FH 運命を笑いまちをり畢業す}\hfill{\FH 同}

    {\FK 贺人毕业}

    {\FK 今日学业遂辞校涉社会运命竟何似含笑且相待}
\end{haiku}

\begin{haiku}
    {\FH 蝶蝶は媚び舞い蜂は怒りとぶ}\hfill{\FH 同}

    {\FK 粉蝶作态舞蜜蜂含怒飞}
\end{haiku}

\begin{haiku}
    {\FH 船にのせて湖をわたしたる牡丹かな}\hfill{\FH 同}

    {\FK 一苇载牡丹容与渡河来}
\end{haiku}

\begin{haiku}
    {\FH 闘志\ruby{尚}{なお}存して春の風を見る}\hfill{\FH 同}

    {\FK 斗志尚坚劲浩然迎春风}
\end{haiku}

\begin{haiku}
    {\FH 北風を沖いて病の家を訪す}\hfill{\FH 同}

    {\FK 为访卧病者冒寒冲北风}
\end{haiku}

\begin{haiku}
    {\FH 一本の\ruby{沈丁}{じんちょう}の香の館かな}\hfill{\FH 同}

    {\FK 一株沈丁在素馨漫轩楹}
\end{haiku}

\begin{haiku}
    {\FH 風ふくや松にふきつけ花芙蓉}\hfill{\FH 同}

    {\FK 北风吹芙蓉飞花附乔松}
\end{haiku}

\begin{haiku}
    {\FH その舟に\ruby{伴}{ともな}いてゆく春の月}\hfill{\FH 同}

    {\FK 春月若有情长伴客舟行}
\end{haiku}

\begin{haiku}
    {\FH 春めくや子供も唄い母もまた}\hfill{\FH 同}

    {\FK 三月春气暖母子共谣歌}
\end{haiku}

\begin{haiku}
    {\FH だまりいし人話し居る夜長かな}\hfill{\FH 同}

    {\FK 万籁今俱寂人语觉夜长}
\end{haiku}

\begin{haiku}
    {\FH 日を仰ぎ水に\ruby{辺}{あた}りし青きかな}\hfill{\FH 同}

    {\FK 仰首沐流晖水边踏青来}
\end{haiku}

\begin{haiku}
    {\FH 一谷を埋む伽藍や鹿の声}\hfill{\FH 同}

    {\FK 伽蓝藏深谷鹿鸣伴晨昏}
\end{haiku}

\begin{haiku}
    {\FH 仲秋や月明かに人老いし}\hfill{\FH 同}

    {\FK 仲秋璧月明无奈人已老}
\end{haiku}

\begin{haiku}
    {\FH 秋風や庵をめぐりて六十歩}\hfill{\FH 同}

    {\FK 环庵六十步凄其尽秋风}
\end{haiku}

\begin{haiku}
    {\FH \ruby{駒}{こま}の鼻ふくれて動く泉かな}\hfill{\FH 同}

    {\FK 近泉心转喜驹鼻鼓且摇}
\end{haiku}

\begin{haiku}
    {\FH 船に乗れば陸情けあり暮の秋}\hfill{\FH 同}

    {\FK 暮秋乘船行平陆系客情}
\end{haiku}

\begin{haiku}
    {\FH 戸を叩く音かともきく芭蕉かな}\hfill{\FH 同}

    {\FK 声似扣柴门芭蕉动秋风}
\end{haiku}

\begin{haiku}
    {\FH 主客閑話ででむし竹を上るなり}\hfill{\FH 同}

    {\FK 蜗牛缘竹上主客闲话中}
\end{haiku}

\begin{haiku}
    {\FH 石のつゆ流れ流るる\ruby{肌}{はだえ}かな}\hfill{\FH 同}

    {\FK 石上纹理现涓涓清露流}
\end{haiku}

\begin{haiku}
    {\FH 旅悲し\ruby{柳絮}{りゅうじょ}とびかうばかりけり}\hfill{\FH 同}

    {\FK 唯见柳絮舞旅情无限愁}
\end{haiku}

\begin{haiku}
    {\FH 慟哭せしは昔となりぬ明治節}\hfill{\FH 同}

    {\FK 史实成陈迹恸哭明治节}
\end{haiku}

\begin{haiku}
    {\FH 君と共に四十年の秋を見し}\hfill{\FH 同}

    {\FK 悼王城}

    {\FK 今日长回首交游四十秋}
\end{haiku}

\begin{haiku}
    {\FH \ruby{志}{こころざし}奪うべからず秋の風}\hfill{\FH 同}

    {\FK 秋风纵疾劲难夺云汉心}
\end{haiku}

\begin{haiku}
    {\FH 各各の薔薇手にして園の出す}\hfill{\FH 同}

    {\FK 蔷薇各在手相偕出园来}
\end{haiku}

\begin{haiku}
    {\FH \ruby{雪解}{ゆきげ}風あたたかしともさむしとも}\hfill{\FH 同}

    {\FK 积雪今融解北风暖亦寒}
\end{haiku}

\begin{haiku}
    {\FH 秋水に解けては結ぶ\ruby{魚紋}{ぎょもん}かな}\hfill{\FH 同}

    {\FK 鱼纹生秋水倏忽聚又开}
\end{haiku}

\begin{haiku}
    {\FH 我為に主婦が座右の蝿を打つ}\hfill{\FH 同}

    {\FK 主妇待我厚扑打座右蝇}
\end{haiku}

\begin{haiku}
    {\FH 満ちてかけかけて満つるや阿蘇の月}\hfill{\FH 同}

    {\FK 祝俳志『阿苏』廿四周年}

    {\FK 盈亏亏又盈月照阿苏花}
\end{haiku}

\begin{haiku}
    {\FH 夏花地に落ちて法の力の衰えじ}\hfill{\FH 同}

    {\FK 悼拙童和尚}

    {\FK 夏花坠平地法力岂式微}
\end{haiku}

\begin{haiku}
    {\FH \ruby{岐}{わか}れ道いくつもありて桑の道}\hfill{\FH 同}

    {\FK 大道多歧路阡陌茂春桑}
\end{haiku}

\begin{haiku}
    {\FH 春風に向いて歩く\ruby{徐}{おもむ}ろに}\hfill{\FH 同}

    {\FK 举步徐徐去披襟向春风}
\end{haiku}

\begin{haiku}
    {\FH 自ら草といい芳草といわず}\hfill{\FH 同}

    {\FK 祝俳志『草』卅五周年}

    {\FK 自度唯小草不肯称瑶芳}
\end{haiku}

\begin{haiku}
    {\FH 親鹿に追ひつきたりし子鹿かな}\hfill{\FH 同}

    {\FK 追随慈亲去幼鹿方振蹄}
\end{haiku}

\begin{haiku}
    {\FH 厚布団薄布団旅つづけけり}\hfill{\FH 同}

    {\FK 布被厚复薄长在逆旅中}
\end{haiku}

\begin{haiku}
    {\FH 涙痕の頰に\ruby{乾}{かわ}きたるひるねかな}\hfill{\FH 同}

    {\FK 唁木国丧女}

    {\FK 夏日人昼卧颊上泪痕干}
\end{haiku}

\begin{haiku}
    {\FH 銀河濃し\ruby{郷愁}{きょうしゅう}なきにしもあらず}\hfill{\FH 同}

    {\FK 银汉色浓甚能不怀乡愁}
\end{haiku}

\begin{haiku}
    {\FH 袖に来て遊び消ゆるや春の雪}\hfill{\FH 同}

    {\FK 来游衣袖里春雪转瞬消}
\end{haiku}

\begin{haiku}
    {\FH \ruby{老衲}{らうなふ}火燵に在り立春の禽獣裏山に}\hfill{\FH 同}

    {\FK 寒冬今已去阳春又重还老衲拥炉坐禽兽藏深山}
\end{haiku}

\begin{haiku}
    {\FH 爛々と昼の星見え菌生え}\hfill{\FH 同}

    {\FK 昼间繁星现闪烁蘑菇生}
\end{haiku}

\begin{haiku}
    {\FH 二三子の\ruby{携}{たずさ}え来る新酒かな}\hfill{\FH 同}

    {\FK 来访二三子新酒提携中}
\end{haiku}

\begin{haiku}
    {\FH 寒月を網する如き枯枝かな}\hfill{\FH 同}

    {\FK 似欲网寒月纵横枯枝多}
\end{haiku}

\begin{haiku}
    {\FH 桐一葉日当たりながら落ちにけり}\hfill{\FH 同}

    {\FK 日光纵灿烂一叶落修桐}
\end{haiku}

\begin{haiku}
    {\FH 秋風や\ruby{顧}{かえり}みずして相別る}\hfill{\FH 同}

    {\FK 前行不回顾分袂秋风中}
\end{haiku}

\begin{haiku}
    {\FH 草庵を菊の館とも誇りけり}\hfill{\FH 同}

    {\FK 草庵何陶然堪夸丛菊馆}
\end{haiku}

\begin{haiku}
    {\FH 亡国の\ruby{狭斜}{きょうしゃ}美し春惜む}\hfill{\FH 同}

    {\FK 春暮成惆怅艳姬倾国容}
\end{haiku}

\begin{haiku}
    {\FH 秋風に\ruby{殖}{ふ}えては減るや法師蝉}\hfill{\FH 同}

    {\FK 玄蝉鸣不住增减随秋风}
\end{haiku}

\begin{haiku}
    {\FH 秋風や眼中のもの皆俳句}\hfill{\FH 同}

    {\FK 秋风含情送诗意万物触目皆篇章}
\end{haiku}

\begin{haiku}
    {\FH 秋の時雨竹の時雨\ruby{賞}{め}でにけり}\hfill{\FH 同}

    {\FK 时雨从来堪叹赏零松洒竹俱有情}
\end{haiku}

\begin{haiku}
    {\FH 盛んなる菊の面影残りけり}\hfill{\FH 同}

    {\FK 阑珊晚香风韵在能不怀想浓艳时}
\end{haiku}

\begin{haiku}
    {\FH 枯菊の色尚存す霜の庭}\hfill{\FH 同}

    {\FK 独怜枯菊经霜后尚含妩媚立庭中}
\end{haiku}

\begin{haiku}
    {\FH 秒針の音心臓の音夜半の冬}\hfill{\FH 同}

    {\FK 隆冬夜半难成寐秒针心搏历历闻}
\end{haiku}

\begin{haiku}
    {\FH 年忘れ老は淋しく笑まひをり}\hfill{\FH 同}

    {\FK 岁晏叙饮忘年酒寂寞衰容笑颜开}
\end{haiku}

\begin{haiku}
    {\FH 昨日も来今日も亦くる紅葉濃し}\hfill{\FH 同}

    {\FK 为爱红叶色如丹昨日去后今又来}
\end{haiku}

\begin{haiku}
    {\FH 花遅し昨日は余寒今日は雷}\hfill{\FH 同}

    {\FK 自是禊花开较晚昨日余寒今日雷}
\end{haiku}

\begin{haiku}
    {\FH 秋蝉も泣き蓑虫も泣くのみぞ}\hfill{\FH 同}

    {\FK 终战}

    {\FK 寰宇从此息征战秋蝉蓑虫尽生哀}
\end{haiku}

\begin{haiku}
    {\FH 草の露三千人の涙かな}\hfill{\FH 同}

    {\FK 哭子规先生}

    {\FK 清秋凉露草上满三千同方哭先生}
\end{haiku}

\begin{haiku}
    {\FH 賊舟を追う兵船や夏の海}\hfill{\FH 同}

    {\FK 宁道夏海烽燧静遥见艨艟逐盗舟}
\end{haiku}

\begin{haiku}
    {\FH \ruby{生涯}{しょうがい}に二度ある悔や秋の風}\hfill{\FH 同}

    {\FK 生涯长留两度恨惆怅今日又秋风}
\end{haiku}

\begin{haiku}
    {\FH 草の戸の残暑というも昨日今日}\hfill{\FH 同}

    {\FK 昨日今日皆苦热蓬窗茅户残暑留}
\end{haiku}

\begin{haiku}
    {\FH 老柳に句碑に名残や今日\ruby{下山}{げざん}}\hfill{\FH 同}

    {\FK 可奈今日下山去老柳句碑牵情多}
\end{haiku}

\begin{haiku}
    {\FH 下闇やならびて皇子の墳二つ}\hfill{\FH 同}

    {\FK 两座殡宫相并立皇子长眠浓荫中}
\end{haiku}

\begin{haiku}
    {\FH \ruby{袷著}{あわせき}て仮の世にある我等かな}\hfill{\FH 同}

    {\FK 悼醉佛}

    {\FK 辞却浮世溘然去长别我等夹衣人}
\end{haiku}

\begin{haiku}
    {\FH 短夜や夢も\ruby{現}{うつつ}も同じこと}\hfill{\FH 同}

    {\FK 短夜匆匆枕边度梦境现实浑相同}
\end{haiku}

\begin{haiku}
    {\FH 雨風にまかせていたむ牡丹かな}\hfill{\FH 同}

    {\FK 惋惜牡丹柔弱质雨横风狂任暴凌}
\end{haiku}

\begin{haiku}
    {\FH 寝し家を喜びとべる蛍かな}\hfill{\FH 同}

    {\FK 流萤陶陶复欣欣只向入梦人家飞}
\end{haiku}

\begin{haiku}
    {\FH \ruby{行水}{ぎょうずい}の女に惚れる烏かな}\hfill{\FH 同}

    {\FK 丽姝入浴太妖娆枝头旅乌亦魂销}
\end{haiku}

\begin{haiku}
    {\FH 美人おつるが如くに椿おちにけり}\hfill{\FH 同}

    {\FK 娇媚山茶梢头落宛如美姬降人间}
\end{haiku}

\begin{haiku}
    {\FH 見送りの人等らを寒きやみにのこし}\hfill{\FH 同}

    {\FK 入夜凜竞寒气重去留人立空濛中}
\end{haiku}

\begin{haiku}
    {\FH 霧の中に簑の人現れ又隠れぬ}\hfill{\FH 同}

    {\FK 乍隐乍现迷离处蓑衣人在雾中行}
\end{haiku}

\begin{haiku}
    {\FH 一生も終り近し梅をみる}\hfill{\FH 同}

    {\FK 自怜蜉蝣行看尽今朝又见梅花开}
\end{haiku}

\begin{haiku}
    {\FH 枯れ枝に初春の雨の玉\ruby{円}{まど}か}\hfill{\FH 同}

    {\FK 初春雨洒枯枝上如玉水珠滴滴圆}
\end{haiku}

\begin{haiku}
    {\FH 春風や闘志いだきて丘に立つ}\hfill{\FH 同}

    {\FK 春风骀荡吹胸襟斗志盈怀立高丘}
\end{haiku}

\begin{haiku}
    {\FH 薔薇剪って短き詩をぞ作りける}\hfill{\FH 同}

    {\FK 修剪蔷薇诗兴动新诗吟哦得几回}
\end{haiku}

\begin{haiku}
    {\FH 北に富士南に我家梅の花}\hfill{\FH 同}

    {\FK 北望富士耸天际吾庐在南梅花开}
\end{haiku}

\begin{haiku}
    {\FH \ruby{謹}{つつし}んで君が\ruby{遺稿}{いこう}を読みはじむ}\hfill{\FH 同}

    {\FK 新岁宜取佳作诵拜读足下遗稿时}
\end{haiku}

\begin{haiku}
    {\FH 霜を掃き山茶花を掃くばかりかな}\hfill{\FH 同}

    {\FK 冬日何计遣寂寞亦扫山茶亦扫霜}
\end{haiku}

\begin{haiku}
    {\FH \ruby{豊年}{ほうねん}の田の面に案山子沈み居り}\hfill{\FH 同}

    {\FK 菽麦茁壮掩草人丰岁喜见年谷登}
\end{haiku}

\begin{haiku}
    {\FH 春雨のふるもはるるも花白し}\hfill{\FH 同}

    {\FK 可爱樱花总洁白任它春雨落还睛}
\end{haiku}

\begin{haiku}
    {\FH 灰の如き記憶ただあり年暮るる}\hfill{\FH 同}

    {\FK 隆冬已至岁又暮缅想旧事剩余灰}
\end{haiku}

\begin{haiku}
    {\FH 一木のかくひろがりししげりかな}\hfill{\FH 同}

    {\FK 枝条荫翳复蓊郁大木成荫广且茂}
\end{haiku}

\begin{haiku}
    {\FH 暑に堪へて双親あるや水を打つ}\hfill{\FH 同}

    {\FK 暑热难耐双亲在殷勤泼水好追凉}
\end{haiku}

\begin{haiku}
    {\FH 秋草の名もなきをわが墓に植えよ}\hfill{\FH 同}

    {\FK 无名秋草素珍重请君植吾幽圹前}
\end{haiku}

\begin{haiku}
    {\FH 海を見て松の落葉の欄に倚る}\hfill{\FH 同}

    {\FK 与子规子在须磨}

    {\FK 松叶摇落遍栏干旅人倚此望海涛}
\end{haiku}

\begin{haiku}
    {\FH 月を思い人を思いて須磨にあり}\hfill{\FH 同}

    {\FK 既念明月复怀人须磨客里旅情多}
\end{haiku}

\begin{haiku}
    {\FH 短夜に似たる四年の月日かな}\hfill{\FH 同}

    {\FK 弹指四年等逝水回首何如一夜长}
\end{haiku}

\begin{haiku}
    {\FH 火の国の筑紫の旅の日焼かな}\hfill{\FH 同}

    {\FK 客途赤日灼肌肤筑紫国似红焰乡}
\end{haiku}

\begin{haiku}
    {\FH 熱帯の海は日を呑み終りたる}\hfill{\FH 同}

    {\FK 热带沧海无涯岸吞咽落日藏碧波}
\end{haiku}

\begin{haiku}
    {\FH 蒲公英の黄が目に残り障子に黄}\hfill{\FH 同}

    {\FK 蒲公英色留眼里转视纸窗亦觉黄}
\end{haiku}

\begin{haiku}
    {\FH 秋風や瀬田の唐橋三百歩}\hfill{\FH 同}

    {\FK 濑田唐挢三百步徘徊只在秋风中}
\end{haiku}

\begin{haiku}
    {\FH 雪折の竹を惜しむに似たるかな}\hfill{\FH 同}

    {\FK 悼渥美溪月}

    {\FK 寒竹折干为雪重长恨如君弃世时}
\end{haiku}

\begin{haiku}
    {\FH 右手は勇左手は仁や\ruby{懐手}{ふところで}}\hfill{\FH 同}

    {\FK 右手勇兮左手仁冬日袖手感怀中}
\end{haiku}

\begin{haiku}
    {\FH 秋重畳俳書万巻の主なし}\hfill{\FH 同}

    {\FK 悼大野洒竹}

    {\FK 万卷俳书任重叠秋来主人何处归}
\end{haiku}

\begin{haiku}
    {\FH 小説に書く女より椿\ruby{艶}{つや}}\hfill{\FH 同}

    {\FK 说部美人虽娇艳枝头山茶更风流}
\end{haiku}

\begin{haiku}
    {\FH 樹齢約四百年の落葉かな}\hfill{\FH 同}

    {\FK 古木逢秋亦落叶树龄已近四百年}
\end{haiku}

\begin{haiku}
    {\FH 別荘にきて\ruby{啓蟄}{けいちつ}の虫を友}\hfill{\FH 同}

    {\FK 启蛰百虫皆吾友仲春三月来别庄}
\end{haiku}

\begin{haiku}
    {\FH 訪いくる雪の山家にある我を}\hfill{\FH 同}

    {\FK 雪封山家我犹在殷勤来访有伊人}
\end{haiku}

\begin{haiku}
    {\FH 他にもあり雨の海棠訪う人は}\hfill{\FH 同}

    {\FK 亦有他人同我兴四月雨中赏海棠}
\end{haiku}

\begin{haiku}
    {\FH 梅を\ruby{探}{さぐ}りて病める老尼に二三言}\hfill{\FH 同}

    {\FK 探梅适逢老尼病随喜交谈二三言}
\end{haiku}

\begin{haiku}
    {\FH 夕立や森を出で来る馬車一つ}\hfill{\FH 同}

    {\FK 凉雨夜降却暑热一驾马车出林来}
\end{haiku}

\begin{haiku}
    {\FH 秋風や相逢はざるも亦よろし}\hfill{\FH 同}

    {\FK 凉秋已至金风起纵不相逢亦欣然}
\end{haiku}

\begin{haiku}
    {\FH 目\ruby{瞑}{つむ}れば若き我あり春の宵}\hfill{\FH 同}

    {\FK 荳蔻年华有我在春宵瞑目沉思中}
\end{haiku}

\begin{haiku}
    {\FH 一本の枯木がくれの帰雁かな}\hfill{\FH 同}

    {\FK 栖隐一株枯木上万里云路旅雁归}
\end{haiku}

\begin{haiku}
    {\FH \ruby{寒雁}{かんがん}の声\ruby{岬}{さき}風に消えにけり}\hfill{\FH 乙字}

    {\FK 寒雁声声唤倶随岬风消}
\end{haiku}

\begin{haiku}
    {\FH 蒲公英蒲公英砂浜に春が目を開く}\hfill{\FH 井泉水}

    {\FK 蒲公英花满砂滨春眼开}
\end{haiku}

\begin{haiku}
    {\FH 詩を狩るとし小鳥の驚かせしか}\hfill{\FH 同}

    {\FK 小鸟惊动否人因觅诗来}
\end{haiku}

\begin{haiku}
    {\FH 天風や雲雀の声を絶つしばし}\hfill{\FH 亜浪}

    {\FK 云雀声暂绝长空过天风}
\end{haiku}

\begin{haiku}
    {\FH 思い遠し月になりゆく第一関}\hfill{\FH 同}

    {\FK 山海关}

    {\FK 异乡为客思绪远月夜来抵第一关}
\end{haiku}

\begin{haiku}
    {\FH 木曽路ゆく我も旅人散る木の葉}\hfill{\FH 同}

    {\FK 落叶洒遍木曾路我身亦是客中人}
\end{haiku}

\begin{haiku}
    {\FH 短夜や嵐忘れし峠宿}\hfill{\FH 六花}

    {\FK 忘却风雨骤短夜宿峰巅}
\end{haiku}

\begin{haiku}
    {\FH 我が\ruby{寡言}{かげん}知る客安き夜長かな}\hfill{\FH 同}

    {\FK 知我寡言语长夜客安然}
\end{haiku}

\begin{haiku}
    {\FH 死期明らかなり山茶花の咲き誇る}\hfill{\FH 一碧楼}

    {\FK 死期分明时已近夸艳争发山茶花}
\end{haiku}

\begin{haiku}
    {\FH 我死ぬ家柿の木ありて花野見ゆ}\hfill{\FH 同}

    {\FK 能眺花野有柿树私愿浮生终此家}
\end{haiku}

\begin{haiku}
    {\FH 病めば蒲団のそと冬海の青きを覚え}\hfill{\FH 同}

    {\FK 沾疾静卧被褥内青翠冬海入我思}
\end{haiku}

\begin{haiku}
    {\FH \ruby{甕}{かめ}の水にしばらくの月寄れり}\hfill{\FH 朱鱗洞}

    {\FK 光照瓮中水良月暂淹留}
\end{haiku}

\begin{haiku}
    {\FH 久し振りの雨の雨だれの音}\hfill{\FH 放哉}

    {\FK 雨露久未润今闻檐溜声}
\end{haiku}

\begin{haiku}
    {\FH まづ耳に入るふるさとの流れなり}\hfill{\FH 鳳車}

    {\FK 何物最先闻故乡流水声}
\end{haiku}

\begin{haiku}
    {\FH \ruby{清談}{せいだん}\ruby{数刻}{すうこく}にして\ruby{罷}{まか}る虫の声}\hfill{\FH 八重桜}

    {\FK 清谈数刻罢凉秋闻虫声}
\end{haiku}

\begin{haiku}
    {\FH 新聞をよむ母の前椿の花を並べた子供}\hfill{\FH 同}

    {\FK 山茶花发儿搬弄并列读报慈母前}
\end{haiku}

\begin{haiku}
    {\FH 林泉の壮大をみる紅葉かな}\hfill{\FH 竹之門}

    {\FK 林泉体势壮更添红叶浓}
\end{haiku}

\begin{haiku}
    {\FH 風に向い行く若葉あかるき山山}\hfill{\FH 同}

    {\FK 迎风放步去山山嫩叶明}
\end{haiku}

\begin{haiku}
    {\FH いつも虫なく父が命日の今年は暑くて}\hfill{\FH 同}

    {\FK 亡父忌辰今年暑以往岁岁闻秋虫}
\end{haiku}

\begin{haiku}
    {\FH 不治病を得て歓喜ありかえり花}\hfill{\FH 同}

    {\FK 虽得绝症亦欢喜一年两度睹花开}
\end{haiku}

\begin{haiku}
    {\FH 君に贈る道遠し画蘭送りけり}\hfill{\FH 師竹}

    {\FK 画兰聊相赠怜君道路长}
\end{haiku}

\begin{haiku}
    {\FH 冬山やどこまで登る\ruby{郵便}{ゆうびん}夫}\hfill{\FH 水巴}

    {\FK 未知行何处邮差上冬山}
\end{haiku}

\begin{haiku}
    {\FH 白日は我が霊なりし落葉かな}\hfill{\FH 同}

    {\FK 白日翻飞坠落叶是我魂}
\end{haiku}

\begin{haiku}
    {\FH \ruby{除夜}{じょや}の灯のどこも人住む野山かな}\hfill{\FH 同}

    {\FK 除夜灯处处山野满人家}
\end{haiku}

\begin{haiku}
    {\FH 大雪や幽明わかず町寝たり}\hfill{\FH 同}

    {\FK 幽明不可辨城眠大雪中}
\end{haiku}

\begin{haiku}
    {\FH 秋風や模様のちがふ皿二つ}\hfill{\FH 石鼎}

    {\FK 两皿秋风里模样浑不同}
\end{haiku}

\begin{haiku}
    {\FH 青天や白き五弁の梨の花}\hfill{\FH 同}

    {\FK 梨花分五瓣洁白映青天}
\end{haiku}

\begin{haiku}
    {\FH 提灯を蛍が襲う谷を来り}\hfill{\FH 同}

    {\FK 入夜来谷里流萤袭灯笼}
\end{haiku}

\begin{haiku}
    {\FH 蔓踏んで一山の露動きけり}\hfill{\FH 同}

    {\FK 为因藤蔓踩足下遂教凉露动一山}
\end{haiku}

\begin{haiku}
    {\FH \ruby{冴}{さ}え返り冴え返りつつ春なかば}\hfill{\FH 泊雲}

    {\FK 阳春虽云半重寒复重寒}
\end{haiku}

\begin{haiku}
    {\FH \ruby{疾}{と}くゆるく露流れ居る木膚かな}\hfill{\FH 同}

    {\FK 木纹似肌肤徐疾清露流}
\end{haiku}

\begin{haiku}
    {\FH \ruby{潦}{にわたずみ}に映りては消ゆ春の雪}\hfill{\FH 同}

    {\FK 飘飘落积水春雪映又消}
\end{haiku}

\begin{haiku}
    {\FH \ruby{昨夜}{よべ}の雨吸いし大地の落花かな}\hfill{\FH 同}

    {\FK 饱吸昨夜雨大地载落花}
\end{haiku}

\begin{haiku}
    {\FH ゆくわれに星も従う水田かな}\hfill{\FH 同}

    {\FK 来步水田里秋星从吾行}
\end{haiku}

\begin{haiku}
    {\FH 灯に\ruby{映}{は}えて金魚赤さや風雨の夜}\hfill{\FH 同}

    {\FK 夜来风雨起灯下金鱼红}
\end{haiku}

\begin{haiku}
    {\FH 春風や我苦言\ruby{容}{い}る君が\ruby{眉宇}{びう}}\hfill{\FH 同}

    {\FK 我道苦情春风里忧思现君眉宇间}
\end{haiku}

\begin{haiku}
    {\FH 風の月壁はなれとぶ\ruby{乾菜}{ほしな}かげ}\hfill{\FH 同}

    {\FK 因风离壁欲飞动干菜影映月明中}
\end{haiku}

\begin{haiku}
    {\FH 蛍火や涙を洗う\ruby{御溝}{みかわ}水}\hfill{\FH 鼠骨}

    {\FK 泪洗御沟水萤火点点明}
\end{haiku}

\begin{haiku}
    {\FH 遠望をさえぎる紅葉一枝かな}\hfill{\FH 泊月}

    {\FK 遮断凝睇目一枝红叶浓}
\end{haiku}

\begin{haiku}
    {\FH 世をわすれ世に忘れられ日向ぼこ}\hfill{\FH 同}

    {\FK 负曝冬日下物我两忘情}
\end{haiku}

\begin{haiku}
    {\FH 圓虹にたちむかいたる岩かな}\hfill{\FH 同}

    {\FK 大岩巍峨立翘然对圆虹}
\end{haiku}

\begin{haiku}
    {\FH 指先を流るる如し種を蒔く}\hfill{\FH 同}

    {\FK 种籽播田里如从指端流}
\end{haiku}

\begin{haiku}
    {\FH 花うてばとびさる蝶の怒りかな}\hfill{\FH 瓦全}

    {\FK 击花又飞去粉蝶怒不休}
\end{haiku}

\begin{haiku}
    {\FH 春暁やかさなりひびく二寺の鐘}\hfill{\FH 同}

    {\FK 春晓声重叠二寺齐鸣钟}
\end{haiku}

\begin{haiku}
    {\FH 元日や凛冽として松の霜}\hfill{\FH 同}

    {\FK 苍松清霜染元旦凛冽寒}
\end{haiku}

\begin{haiku}
    {\FH \ruby{憂}{う}きことも去年になりゆく懐しや}\hfill{\FH 同}

    {\FK 而今仍怀旧忧思成去年}
\end{haiku}

\begin{haiku}
    {\FH 大内山松の上なる\ruby{淑気}{しゅくき}かな}\hfill{\FH 同}

    {\FK 大内山耸万松上春到人间淑气浓}
\end{haiku}

\begin{haiku}
    {\FH 春尽きて山みな甲斐に走りけり}\hfill{\FH 普羅}

    {\FK 春光今已尽群山聚甲斐}
\end{haiku}

\begin{haiku}
    {\FH 雪解川名山けづる響かな}\hfill{\FH 同}

    {\FK 长河积雪解声震响名山}
\end{haiku}

\begin{haiku}
    {\FH 旅人は休まずありく落葉の香}\hfill{\FH 同}

    {\FK 旅人不住步落叶送芬芳}
\end{haiku}

\begin{haiku}
    {\FH 淋しさや春山を描き雲を添ふ}\hfill{\FH 同}

    {\FK 伶仃无所适添云画春山}
\end{haiku}

\begin{haiku}
    {\FH 秋風の吹きくる方に帰るなり}\hfill{\FH 同}

    {\FK 秋风来何所愿向彼处归}
\end{haiku}

\begin{haiku}
    {\FH 春の夜や粧ひ終へし蝋短か}\hfill{\FH 久女}

    {\FK 春夜蜡炬短脂粉却扫中}
\end{haiku}

\begin{haiku}
    {\FH \ruby{常}{とこ}夏の碧き潮あびわが\ruby{育}{そだ}つ}\hfill{\FH 同}

    {\FK 此是育我地长夏浴碧潮}
\end{haiku}

\begin{haiku}
    {\FH 谺して山ほととぎすほしいまま}\hfill{\FH 同}

    {\FK 英彦山}

    {\FK 山中时鸟悠闲甚空谷清音传回声}
\end{haiku}

\begin{haiku}
    {\FH 魂のたとえば秋の蛍かな}\hfill{\FH 蛇笏}

    {\FK 悼芥川龙之介}

    {\FK 精魂何所似秋萤差可拟}
\end{haiku}

\begin{haiku}
    {\FH 秋の風死して世を視る細眼なほ}\hfill{\FH 同}

    {\FK 细眼犹视世长逝秋风中}
\end{haiku}

\begin{haiku}
    {\FH 炎天を\ruby{槍}{やり}のごとくに\ruby{涼気}{すずけ}すぐ}\hfill{\FH 同}

    {\FK 炎天凉气过直行锐如枪}
\end{haiku}

\begin{haiku}
    {\FH 暖かく掃きし墓前を去りがたし}\hfill{\FH 同}

    {\FK 吊长男鸥生}

    {\FK 绕坟不忍去春暖扫墓前}
\end{haiku}

\begin{haiku}
    {\FH 痩せし身の眼の生きるのみ秋の霜}\hfill{\FH 同}

    {\FK 自画像}

    {\FK 瘦躯立秋霜唯有目生光}
\end{haiku}

\begin{haiku}
    {\FH 春雪に子の死あひつぐ朝の燭}\hfill{\FH 同}

    {\FK 三男死去}

    {\FK 亡儿讣报频仍至蜡炬犹明春雪朝}
\end{haiku}

\begin{haiku}
    {\FH 星屑や鬱然として夜の新樹}\hfill{\FH 草城}

    {\FK 长空星屑满春树夜郁然}
\end{haiku}

\begin{haiku}
    {\FH 大風にはげしく匂う新樹かな}\hfill{\FH 同}

    {\FK 入夏新树绿浓香大风中}
\end{haiku}

\begin{haiku}
    {\FH 春暁や人こそ知らね木々の雨}\hfill{\FH 同}

    {\FK 万木皆滴露春晓人不知}
\end{haiku}

\begin{haiku}
    {\FH 木枯や翠も暗き東山}\hfill{\FH 同}

    {\FK 翠山亦暗淡东山树尽凋}
\end{haiku}

\begin{haiku}
    {\FH 妻子を\ruby{担}{かる}ふ片眼片肺枯れ手足}\hfill{\FH 同}

    {\FK 自励}

    {\FK 赡养妻儿负重荷片眼独肺手足枯}
\end{haiku}

\begin{haiku}
    {\FH 妻も覚めてすこし話や夜半の春}\hfill{\FH 同}

    {\FK 荆妻觉来聊共语春宵骎骎夜未央}
\end{haiku}

\begin{haiku}
    {\FH 老松のにぎわい立てる翠かな}\hfill{\FH 風生}

    {\FK 映眼春意闹老松生绿芽}
\end{haiku}

\begin{haiku}
    {\FH 恩を謝し怨を忘れあたたかし}\hfill{\FH 同}

    {\FK 暖意春盎然谢恩忘怨仇}
\end{haiku}

\begin{haiku}
    {\FH 大寒と敵のごとく対ひたり}\hfill{\FH 同}

    {\FK 严阵如对敌大寒今已来}
\end{haiku}

\begin{haiku}
    {\FH 読む窓に灯よりも\ruby{淡}{あわ}く秋日影}\hfill{\FH 同}

    {\FK 书窗秋日影淡于入夜灯}
\end{haiku}

\begin{haiku}
    {\FH しおしおと\ruby{杏花}{きょうか}ぼのぼのと旅情}\hfill{\FH 同}

    {\FK 隐约旅情重萧寂杏花幵}
\end{haiku}

\begin{haiku}
    {\FH つんつんと風の\ruby{躑躅}{つつじ}の\ruby{蕾}{つぼみ}立ち}\hfill{\FH 同}

    {\FK 踯躅花蕾挺含骄立春风}
\end{haiku}

\begin{haiku}
    {\FH 諸子庭に春の火桶に主わが}\hfill{\FH 同}

    {\FK 春日吾为火钵主诸子皆在庭院中}
\end{haiku}

\begin{haiku}
    {\FH \ruby{久闊}{きゅうかつ}を\ruby{叙}{じょ}し椋鳥を共に仰ぐ}\hfill{\FH 同}

    {\FK 久别情怀今畅叙仰首共望白头翁}
\end{haiku}

\begin{haiku}
    {\FH 咳をする母を見上げてゐる子かな}\hfill{\FH 汀女}

    {\FK 仰面望慈亲母晐儿惊心}
\end{haiku}

\begin{haiku}
    {\FH 咳の子のなぞなぞ遊びきりもなや}\hfill{\FH 同}

    {\FK 长伴咳嗽子猜谜无时休}
\end{haiku}

\begin{haiku}
    {\FH 入学児胸呑ませ穿く\ruby{長袴}{ながばかま}}\hfill{\FH 青畝}

    {\FK 小儿入学去长裤几没胸}
\end{haiku}

\begin{haiku}
    {\FH 秋の蝶われを\ruby{且}{かつ}追い且迎う}\hfill{\FH 同}

    {\FK 秋蝶何欣欣追我又相迎}
\end{haiku}

\begin{haiku}
    {\FH 寒月の通天わたるひとりかな}\hfill{\FH 茅舎}

    {\FK 长天踪迹遍寒月独遨游}
\end{haiku}

\begin{haiku}
    {\FH 洞然と雷聞きて\ruby{未}{ま}だ生きて}\hfill{\FH 同}

    {\FK 响亮闻雷震人世犹弥留}
\end{haiku}

\begin{haiku}
    {\FH 夏瘦せて腕は鐵棒より重し}\hfill{\FH 同}

    {\FK 腕重胜铁棒憔悴长夏中}
\end{haiku}

\begin{haiku}
    {\FH うつくしきかなしき話秋の濱}\hfill{\FH 青邨}

    {\FK 野岸旁秋水悲喜话从头}
\end{haiku}

\begin{haiku}
    {\FH 流れ入り流れ去る水苔の花}\hfill{\FH 同}

    {\FK 流水出复入苔花载沉浮}
\end{haiku}

\begin{haiku}
    {\FH 人それぞれ書を読んでゐる良夜かな}\hfill{\FH 同}

    {\FK 捧书各成诵良夜月华明}
\end{haiku}

\begin{haiku}
    {\FH 春蘭や雑木未だ日を遮らず}\hfill{\FH 同}

    {\FK 杂木未遮日光尽振振春兰自茁生}
\end{haiku}

\begin{haiku}
    {\FH 蒲公英や長江濁るとこしなへ}\hfill{\FH 同}

    {\FK 长江一往千古浊蒲公英色自鲜明}
\end{haiku}

\begin{haiku}
    {\FH つゆの世の齢六十にしてたのし}\hfill{\FH 同}

    {\FK 浮生须臾同朝露行年六十喜开怀}
\end{haiku}

\begin{haiku}
    {\FH 菊の香の夜の扉に合掌す}\hfill{\FH 素十}

    {\FK 菊前谨合掌幽香绕夜扉}
\end{haiku}

\begin{haiku}
    {\FH 秋雨と秋風をきく枕かな}\hfill{\FH 同}

    {\FK 枕上何所闻秋雨复秋风}
\end{haiku}

\begin{haiku}
    {\FH 旅人や枯野の月を仰ぎけり}\hfill{\FH 同}

    {\FK 旅人翘首望枯野夜月明}
\end{haiku}

\begin{haiku}
    {\FH 青桐の向うの家の煙出し}\hfill{\FH 同}

    {\FK 冉冉炊烟起人家对青桐}
\end{haiku}

\begin{haiku}
    {\FH 春泥に押しあひながら来る娘}\hfill{\FH 同}

    {\FK 少女杂沓至推挤步春泥}
\end{haiku}

\begin{haiku}
    {\FH 夢さめておどろく闇や秋の暮}\hfill{\FH 秋桜子}

    {\FK 梦醒惊幽暗秋日近黄昏}
\end{haiku}

\begin{haiku}
    {\FH 野の虹と春田の虹と空に合ふ}\hfill{\FH 同}

    {\FK 原野春田虹各现两方交会长空中}
\end{haiku}

\begin{haiku}
    {\FH 学問のさびしさに堪え\ruby{炭}{すみ}をつぐ}\hfill{\FH 誓子}

    {\FK 求学耐寂寞添炭炉火中}
\end{haiku}

\begin{haiku}
    {\FH 七月の\ruby{青嶺}{あおね}まぢかく\ruby{熔鑛炉}{ようこうろ}}\hfill{\FH 同}

    {\FK 七月青岭近巍巍熔矿炉}
\end{haiku}

\begin{haiku}
    {\FH 主の前の日焼童に\ruby{聖寵}{せいちょう}あれ}\hfill{\FH 同}

    {\FK 儿童蒙圣宠神前夕阳红}
\end{haiku}

\begin{haiku}
    {\FH ひとり膝を抱けば秋風また秋風}\hfill{\FH 同}

    {\FK 独自抱膝坐秋风还秋风}
\end{haiku}

\begin{haiku}
    {\FH 露の\ruby{花圃}{かほ}\ruby{天主}{デウス}を祈るものきたる}\hfill{\FH 同}

    {\FK 主前来祈祷花圃凉露溥}
\end{haiku}

\begin{haiku}
    {\FH 炎天の遠き帆やわがこころの帆}\hfill{\FH 同}

    {\FK 炎天帆影远心帆随动摇}
\end{haiku}

\begin{haiku}
    {\FH 臼を\ruby{碾}{ひ}きやみし寒夜の底知れず}\hfill{\FH 同}

    {\FK 寒夜底难测杵臼声今消}
\end{haiku}

\begin{haiku}
    {\FH もの言わず身を秋風につつまるる}\hfill{\FH 同}

    {\FK 默然无言语秋风围我身}
\end{haiku}

\begin{haiku}
    {\FH 蚊帳の月あまりあかるく寝返るか}\hfill{\FH 同}

    {\FK 月魄皎洁照蚊帐人不成眠反侧中}
\end{haiku}

\begin{haiku}
    {\FH いづくにも虹のかけらを拾い得ず}\hfill{\FH 同}

    {\FK 无处可拾碎屑起长天艳丽彩霞生}
\end{haiku}

\begin{haiku}
    {\FH 熱砂走るひびき少女の重さだけ}\hfill{\FH 同}

    {\FK 唯觉少女体沉重步履传声走热砂}
\end{haiku}

\begin{haiku}
    {\FH \ruby{紙漉女}{かみすきめ}頰に\ruby{天賦}{てんぷ}の紅がさす}\hfill{\FH 静塔}

    {\FK 女工正抄纸天賦双颊红}
\end{haiku}

\begin{haiku}
    {\FH 青空はどこへも逃げぬ炭を焼く}\hfill{\FH 同}

    {\FK 青空无匿地烧炭处处烟}
\end{haiku}

\begin{haiku}
    {\FH 母がおくる赤き扇のうれしき風}\hfill{\FH 草田男}

    {\FK 母赐朱红扇摇拂生好风}
\end{haiku}

\begin{haiku}
    {\FH 万緑の中や吾子の歯生え初むる}\hfill{\FH 同}

    {\FK 万物皆绿意吾儿齿初生}
\end{haiku}

\begin{haiku}
    {\FH 蛍火や白き夜道も行路難}\hfill{\FH 同}

    {\FK 萤火夜道白依然行路难}
\end{haiku}

\begin{haiku}
    {\FH 秋の航一大紺圓盤の中}\hfill{\FH 同}

    {\FK 如在大蓝圆盘内客舟破浪作秋航}
\end{haiku}

\begin{haiku}
    {\FH ひとつ見えて秋燈獄に近よらず}\hfill{\FH 不死男}

    {\FK 囹圄不可近独望秋灯遥}
\end{haiku}

\begin{haiku}
    {\FH 降る雪に胸飾られて\ruby{捕}{とら}えらる}\hfill{\FH 同}

    {\FK 见被执送去飞雪饰胸前}
\end{haiku}

\begin{haiku}
    {\FH 手を垂れし影がわれ見る壁寒し}\hfill{\FH 同}

    {\FK 墙映垂手影望壁觉寒生}
\end{haiku}

\begin{haiku}
    {\FH 独房に林檎と寝たる誕生日}\hfill{\FH 同}

    {\FK 生日独抱苹果卧身在单人牢房中}
\end{haiku}

\begin{haiku}
    {\FH 水枕ガバリと寒い海がある}\hfill{\FH 三鬼}

    {\FK 寒海似近我水枕轧轹鸣}
\end{haiku}

\begin{haiku}
    {\FH 算術の少年しのび泣けり夏}\hfill{\FH 同}

    {\FK 少年正为算术苦夏日呑声饮泣中}
\end{haiku}

\begin{haiku}
    {\FH 人遠く春三日月と死が近し}\hfill{\FH 同}

    {\FK 新月春夜死期近寂寞病身人厌弃}
\end{haiku}

\begin{haiku}
    {\FH 秋の夜オリオン低し胸の上}\hfill{\FH 波郷}

    {\FK 沉沉似觉垂胸上秋夜猎户星座低}
\end{haiku}

\begin{haiku}
    {\FH バスを待ち大路の春をうたがはず}\hfill{\FH 同}

    {\FK 静候公共汽车至大路春光何用疑}
\end{haiku}

\begin{haiku}
    {\FH 泉への道\ruby{後}{おく}れゆく安けさよ}\hfill{\FH 同}

    {\FK 此道前行向泉去怡然缓步甘后人}
\end{haiku}

\begin{haiku}
    {\FH 冬帽を脱ぐや蒼茫たる夜空}\hfill{\FH 楸邨}

    {\FK 脱却御寒帽夜空何苍茫}
\end{haiku}

\begin{haiku}
    {\FH 学問の黄昏さむく物を言はず}\hfill{\FH 同}

    {\FK 悄然无言语求学黄昏寒}
\end{haiku}

\begin{haiku}
    {\FH 長き長き春暁の\ruby{貨車}{かしゃ}なつかしき}\hfill{\FH 同}

    {\FK 长列货车过春哓怀儿时}
\end{haiku}

\begin{haiku}
    {\FH 秋晴や歩きゆるめつ園に入る}\hfill{\FH 孝}

    {\FK 秋晴宜漫步缓缓入园来}
\end{haiku}

\begin{haiku}
    {\FH 月光の走れる杖を\ruby{運}{はこ}びけり}\hfill{\FH 同}

    {\FK 策杖为助步扶老走月光}
\end{haiku}

\begin{haiku}
    {\FH 宵闇に\ruby{漁火}{ぎょか}\ruby{鶴翼}{かくよく}の陣を張り}\hfill{\FH 同}

    {\FK 张阵如鹤翼暗夜渔火明}
\end{haiku}

\begin{haiku}
    {\FH 春潮の彼処に怒り此処に笑む}\hfill{\FH 同}

    {\FK 泛荡春潮何窈窕彼处怀嗔此含笑}
\end{haiku}

\begin{haiku}
    {\FH \ruby{草餅}{くさもち}や古る山里の\ruby{言}{こと}\ruby{雅}{みや}び}\hfill{\FH 同}

    {\FK 草饼时节客中度言谈风雅古山家}
\end{haiku}

\begin{haiku}
    {\FH 月光をありがたしとも悲しとも}\hfill{\FH 鮎女}

    {\FK 月华频兴感悲喜从中来}
\end{haiku}

\begin{haiku}
    {\FH またひとつ病重なて秋すすむ}\hfill{\FH 同}

    {\FK 此生长寂寞病重况秋深}
\end{haiku}

\begin{haiku}
    {\FH 寒の雨青き筧を奏でしめ}\hfill{\FH 潮原満}

    {\FK 点点寒雨落青笕奏宫商}
\end{haiku}

\begin{haiku}
    {\FH 大玻璃窓を\ruby{拭}{ふ}き秋空を拭いてをり}\hfill{\FH 上野泰}

    {\FK 拂拭大玻璃如在拭秋空}
\end{haiku}

\begin{haiku}
    {\FH 雲の峰少年の夢かぎりなし}\hfill{\FH 同}

    {\FK 少年梦无限云峰变幻多}
\end{haiku}

\begin{haiku}
    {\FH 水又も月の破片を集めたり}\hfill{\FH 同}

    {\FK 收拾碎月起水波荡漾中}
\end{haiku}

\begin{haiku}
    {\FH 赤富士の天に小鳥の\ruby{微塵}{みじん}かな}\hfill{\FH 同}

    {\FK 赤色富士摩霄立翱翔小鸟若微尘}
\end{haiku}

\begin{haiku}
    {\FH 春月や雫の如火漁火が}\hfill{\FH 朱鳥}

    {\FK 渔火似滴露溶溶春月中}
\end{haiku}

\begin{haiku}
    {\FH 美しき故短命か朝顔は}\hfill{\FH 暁雨}

    {\FK 似同红粉多薄命卷蔓牵牛不终朝}
\end{haiku}

\begin{haiku}
    {\FH 久闊や秋水となり流れゐし}\hfill{\FH 立子}

    {\FK 阔别岁月久合如秋水流}
\end{haiku}

\begin{haiku}
    {\FH 惜春や思ひ出の糸\ruby{縺}{もつ}れ解け}\hfill{\FH 同}

    {\FK 回忆有如细丝线一春尽时缠又开}
\end{haiku}

\begin{haiku}
    {\FH 老いぬれば死の話さえさわやかに}\hfill{\FH 朱城}

    {\FK 老朽何足道爽然话死生}
\end{haiku}

\begin{haiku}
    {\FH 猫耳をたつつ炉話きく如く}\hfill{\FH 同}

    {\FK 冬日围炉人闲话猫竖双耳若谛听}
\end{haiku}

\begin{haiku}
    {\FH \ruby{彫}{え}りふかく\ruby{轍}{てつ}の跡の寒に入る}\hfill{\FH 巴山}

    {\FK 寒气来侵入镂地车辙深}
\end{haiku}

\begin{haiku}
    {\FH 春光や放ちし鳩を呼ぶ口笛}\hfill{\FH 同}

    {\FK 口哨声传春光里放却驯鸠又呼回}
\end{haiku}

\begin{haiku}
    {\FH 年年にかわるゆめおい星祭る}\hfill{\FH 初絵}

    {\FK 年年此夕祭牛女情随时迁梦不同}
\end{haiku}

\begin{haiku}
    {\FH 昨日見し花今日はなし寒椿}\hfill{\FH 夢豪}

    {\FK 悼友}

    {\FK 寒椿今朝艳色尽昨日枝头犹见花}
\end{haiku}

\begin{haiku}
    {\FH \ruby{松籟}{しょうらい}か時雨か院の障子開け}\hfill{\FH 樟子}

    {\FK 向院纸窗今开启来辨松籁时雨声}
\end{haiku}

\begin{haiku}
    {\FH 烈日に切る一塊の氷かな}\hfill{\FH 周平}

    {\FK 一块寒冰解烈日具锐锋}
\end{haiku}

\begin{haiku}
    {\FH 血を吐けば現も夢も冴え返る}\hfill{\FH 寸七翁}

    {\FK 临终}

    {\FK 沉疴不起吐鲜血现实梦境彻骨寒}
\end{haiku}

\begin{haiku}
    {\FH 春江や漕ぎまがりたる長\ruby{筏}{いかだ}}\hfill{\FH 三昧}

    {\FK 沿溯曲复直长筏浮春江}
\end{haiku}

\begin{haiku}
    {\FH 春風や仏を刻む\ruby{鉋}{かんな}\ruby{屑}{くず}}\hfill{\FH 句仏}

    {\FK 能工刻佛像刨屑散春风}
\end{haiku}

\begin{haiku}
    {\FH 秋風やないてばかりも居られざる}\hfill{\FH 雄美}

    {\FK 老妻病没}

    {\FK 泉路九重何所望唯将老泪洒秋风}
\end{haiku}

\begin{haiku}
    {\FH 旅に寝て故郷の春を惜みけり}\hfill{\FH 春武}

    {\FK 故园春光行看尽三更客梦自飞沉}
\end{haiku}

\begin{haiku}
    {\FH オアシスや沙漠の島は葵咲く}\hfill{\FH 楠窗}

    {\FK 绿洲堪称沙漠岛黄尘底里葵花开}
\end{haiku}

\begin{haiku}
    {\FH ネクタイという\ruby{枷}{かし}をより夕\ruby{端居}{はしい}}\hfill{\FH 羽村}

    {\FK 黄昏解脱领带坐宛如除却颈头枷}
\end{haiku}

\begin{haiku}
    {\FH 虫なけばなかねば更に夜のさびし}\hfill{\FH 繞石}

    {\FK 秋夜更寂静虫声任有无}
\end{haiku}

\begin{haiku}
    {\FH 酒の染涙の染や秋袷}\hfill{\FH 孝}

    {\FK 酒痕泪迹满秋袷风尘多}
\end{haiku}

\begin{haiku}
    {\FH 一巻の\ruby{雪嶺}{せつれい}絵巻大玻璃戸}\hfill{\FH 楓石}

    {\FK 大玻璃窗眺远景雪封岭峦绘卷开}
\end{haiku}

\begin{haiku}
    {\FH 目の犬に従き行く盲秋の風}\hfill{\FH 比呂武}

    {\FK 相随举步犬作目失明人行秋风中}
\end{haiku}

\begin{haiku}
    {\FH 笛涼し音色かわりの指はねて}\hfill{\FH 句坊子}

    {\FK 音色变更由指弄三秋闻笛觉声凉}
\end{haiku}

\begin{haiku}
    {\FH 秋風や褒めても\ruby{叱}{し}っても\ruby{呉}{く}れず}\hfill{\FH 安住敦}

    {\FK 悼师久保田万太郎}

    {\FK 褒叱不可闻萧瑟来秋风}
\end{haiku}

\begin{haiku}
    {\FH 炭塵の顏ひとすじの汗の跡}\hfill{\FH 寒子房}

    {\FK 炭尘满颜面一道汗迹流}
\end{haiku}

\begin{haiku}
    {\FH 汗を拭き\ruby{坑}{こう}外のこと話しけり}\hfill{\FH 同}

    {\FK 共话坑外事拭汗憩息中}
\end{haiku}

\begin{haiku}
    {\FH 秋灯下坑内という社会あり}\hfill{\FH 同}

    {\FK 亦然自成一社会秋灯之下有矿坑}
\end{haiku}

\begin{haiku}
    {\FH 夕焼や山をひかへて大\ruby{牧場}{まきば}}\hfill{\FH 杢生}

    {\FK 夕阳红艳堪极目背山轩敞大牧场}
\end{haiku}

\begin{haiku}
    {\FH スキ一帽脱けばおとめの髪溢る}\hfill{\FH 青萍}

    {\FK 少女脱却滑雪帽满头青丝四溢时}
\end{haiku}

\begin{haiku}
    {\FH 月光や大きくのこる君の影}\hfill{\FH 普提子}

    {\FK 送别}

    {\FK 月光清如水庞然君影留}
\end{haiku}

\begin{haiku}
    {\FH 秋耕や蝶来て遊ぶ鍬の先}\hfill{\FH 雨意}

    {\FK 秋耕垅亩里飞蝶游锄尖}
\end{haiku}

\begin{haiku}
    {\FH 春泥や盲の杖にある力}\hfill{\FH 野笛}

    {\FK 赖有竹杖扶持力盲人踯躅春泥中}
\end{haiku}

\begin{haiku}
    {\FH スコ一ルにをどり出でたる裸の子}\hfill{\FH 雨城}

    {\FK 白雨涤荡烦暑尽儿童裸衣雀跃中}
\end{haiku}

\begin{haiku}
    {\FH 読みふける春日縁に\ruby{俯}{うつぶせ}に}\hfill{\FH 米佐子}

    {\FK 俯身阅卷忘所以檐下春阳普照中}
\end{haiku}

\begin{haiku}
    {\FH 鳥に負けて鳶は遠くへ春の空}\hfill{\FH 楽子}

    {\FK 纸鸢遥入春空去输他翔鸟自在飞}
\end{haiku}

\begin{haiku}
    {\FH 秋風や生死わからず便りなぐ}\hfill{\FH 東西}

    {\FK 雁书不随秋风至教从何处问死生}
\end{haiku}

\begin{haiku}
    {\FH 朝顔に日高くなりし無職かな}\hfill{\FH 水雄}

    {\FK 闲身无职如云水静看牵牛日影高}
\end{haiku}

\begin{haiku}
    {\FH 一葉落ちいくらも落ちて月夜かな}\hfill{\FH 嵐雪}

    {\FK 万叶竞随一叶落新秋溶溶夜月中}
\end{haiku}

\begin{haiku}
    {\FH 去る人もとまる人も喜雨に佇ち}\hfill{\FH 沙美}

    {\FK 旱重喜见夏雨降去留人立空濛中}
\end{haiku}

\begin{haiku}
    {\FH 賀客みな吾より若し庵まもる}\hfill{\FH 同}

    {\FK 贺客皆少我独长新岁守庵乐融融}
\end{haiku}
